\section{Preliminary exploration}

In this section, I implemented a simplified completed workflow based 
on the methodology described in Section Methodology. 
It explored the practical application of the proposed method to support the later composite workflow.
Specifically, the major preliminary steps are as follows:

\begin{enumerate}

\item \textbf{Collect comparable data.}

There are three available datasets, containing the three raw data types: zoobenthic community data (311×16), chemical data (104×30), and environmental data (289×7).
The three shared the same identical index - StationID, which supports data merging to prepare completed data for 
each site. After merging there is a combined dataset with 104 rows and 53 columns, containing all three types of data.

\textbf{Key Results:} \(\left[
\begin{array}{ccc}
X & E & T
\end{array}
\right]
\in
\mathbb{R}^{m \times (30 + 5 + 16)}\) was prepared
\footnote{
The column numbers were not consistent(\(51 \neq 53\))
because StationID and location information were not used in the analysis.
Later, the location information can be included to support spatial analysis.
}.

\item \textbf{Assess sediment contamination and Identify reference and degraded sites.}

A log-transformation was applied to the chemical data to reduce dominance by high-value variables.
Then PCA was performed on the transformed chemical data and several principal components were selected to explain
the major variation of pollutant elements. By simply standardizing and summing these PCs, comprehensive stress values were computed for each site.
To keep consistency with Jian's analysis, I used the name "SumReal" to refer these stress values (levels).
Current statistics results shown "the higher are the stress scores, the less are the pollutant elements concentrations",
but this will be further explored in the future.

Based on the computed stress scores,  \(p\%\) was temporarily set to 20\% to identify reference sites and the degraded sites were symmetrically defined.
Out of assumption, there were no or minimal human disturbances on the reference sites, their taxa composition was shaped by environmental conditions only.


\textbf{Key Results:} \(\left[
\begin{array}{ccccc}
X & E & T & s & I_{ref}
\end{array}
\right]
\in
\mathbb{R}^{m \times (51 + 2)}\), \(\) was prepared

\item \textbf{Cluster reference sites by taxa community composition.}

Turn to the zoobenthic community data of these references, IQR method was applied to detect outliers
and then octave transformation was applied to reduce extreme values' impact, 
considering that taxa in low abundance do not mean they are not important.

Then clustering was applied to identify major taxa community patterns across different environmental conditions,
and \(K\) clusters were identified. Here \(K\) was set to 3 through the hierarchical clustering method.
These sites assigned into each \(k\) cluster represent "ideal" taxa structures,
each totally shaped by the range of environmental conditions of the corresponding cluster.

\textbf{Key Results:} 
\(\mathcal{C}_K\)
and
\(\left[
\begin{array}{cccccc}
X & E & T & s & I_{ref} & \mathcal{C}_K
\end{array}
\right]_{I_{ref} = 1}
\in
\mathbb{R}^{(p\% \times m) \times (53 + 1)}\), were prepared.

\item \textbf{Fit Discriminant Function of environmental factors for taxa clusters.}

Based on the identified clusters of these reference sites,
a discriminant function was fitted to predict the taxa cluster membership of each site based on its environmental variables.
\footnote{It is both acceptable to say "environmental clusters" or "taxa clusters", owing to the assumption that 
these reference sites have their taxa composition totally shaped by the environmental conditions.} 

\textbf{Key Results:} 
\(\mathcal{F}_{dis}: e^{(1, 5)} \to \hat C_K\) was fitted.

\item \textbf{Apply the Discriminant Function to rest disturbed sites to group them}

To the (1 - \(p\%\)) disturbed sites, including the degraded sites, 
\(\mathcal{F}_{dis}\) was applied to assign each site to one of the identified taxa clusters.

\textbf{Key Results:} 
\(\mathcal{F}_{dis}: e^{(1, 5)} \to \hat C_K\) was applied, 
\(\left[
\begin{array}{cccccc}
X & E & T & s & I_{ref} & \mathcal{\hat C}_K
\end{array}
\right]_{I_{ref} = 0}\) was prepared.

\item \textbf{Construct endpoints and compute ZCI in each cluster.}

Within each cluster, the reference sites and degraded sites were used to construct endpoints, 
unlike the Multivariate Gaussian cloud of reference sites, 
the endpoints were simply computed as the means of 3 selected references and 3 degraded sites.
The endpoints were used to numerically scale the taxa community structure and compute the Zoobenthic Condition Index via 
Bray-Curtis ordination method. The \(ZCI_{k}\) reflected the distance of the disturbed sites to the endpoints in cluster \(k\).

\textbf{Key Results:} 
\(ZCI_{k}, (k \in {1, ..., K})\) was computed; \(ZCI_{k, j}\) reflects the distance of site \(j\) to its cluster \(k\) endpoints.

\item \textbf{Evaluate the ZCI vs SumReal relationship by quantile regression.}

On top of the \(ZCI\) and stress level \(s\), a piecewise quantile regression was fitted to evaluate their relationship across 
the three taxa clusters. The breakpoints were found by grid search method with a pre-defined searching range, which 
was set manually based on the preliminary exploration.

\textbf{Key Results:} 
\(f_{k, \tau}(z), (k \in {1, ..., K})\) was fitted; \(\hat \theta_{\tau}^{(k)}\) was solved for each cluster \(k\).

\item \textbf{Preliminary exploration of spatial structure with simulation}

An isolated PCNM (spatial eigenfunction) simulation was run, separate from previous workflow steps, to illustrate extraction of spatial structure across increasing complexity. Three scenarios were simulated (low to higher complexity): (i) equally spaced 1D sites, (ii) equally spaced 2D lattice, and (iii) irregular 2D sites with two clusters. The first two yield regular distance matrices and smoothly ordered (broad- to fine-scale) eigenvectors; the clustered irregular layout produces wave-like spatial patterns in the leading eigenfunctions.

\textbf{Key Results:} \(PCNM: D \to S_{\text{sel}}\) was carried out; \(S_{\text{sel}}\) comprises the Moran's I–screened spatial eigenvectors derived from the truncated neighbour distance matrix \(D\).

\item \textbf{Power and sensitivity analysis on quantile regression with simulation}

A deterministic piecewise quantile regression model with fixed breakpoint \(\kappa\) and heteroscedastic noise was used to generate synthetic data. Subsamples of sizes \(n=20,50,80,\dots,500\) were drawn to assess estimation precision of key parameters (\(\hat y_{\tau}\), \(\beta_{1,\tau}\), \(\delta_{\tau}\)). Bootstrap 95\% CIs and precision curves (RMSE) were computed. Power analysis for detecting non-zero hinge effects (\(\delta\neq 0\)) was conducted via repeated simulation under alternative and null conditions.

\textbf{Key Results:} Preliminary patterns show (i) diminishing RMSE gains beyond moderate \(n\); (ii) wider CIs and reduced power at extreme quantiles; (iii) hinge (\(\delta_{\tau}\)) estimates stabilising more slowly than primary slope estimates. Several implementation and interpretation issues (e.g., behaviour at high \(\tau\) with small \(n\)) were noted for later, more detailed exploration.


\end{enumerate}
