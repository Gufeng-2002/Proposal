\section{Introduction}

% \textcolor{blue}{(Background about the study area)}

The Great Lakes, which occupy 84\% 
of North America's surface fresh water and 21\% of the world's supply of surface fresh water
\cite{EPA_Greatlakes2024}, are one of the world's largest surface freshwater ecosystems. 
Many nutrients and contaminants in the Great Lakes are stored in the sediments, playing
a crucial role in supporting aquatic habitats and influencing water quality.

Contaminants in the sediments are, however, unavoidable.
Among the regions, heavily developed shorelines are found around Lake Erie
and the connecting channels such as the Detroit River,
where human activities have left pronounced environmental footprints~\cite{EPA_SOGL2007}. 
These impacts have raised public concerns about the potential ecological risks, 
and have prompted calls for stronger protective and restorative actions 
to safeguard these aquatic ecosystems.

In light of these challenges, scientific assessment of aquatic condition becomes fundamental 
to retaining the integrity of an ecosystem. 
One effective approach is \textbf{to assess the nature and extent of sediment contamination}, 
which directly reflects anthropogenic impacts and serves as a partial indicator of ecological integrity.

This rationale is grounded in a well-established body of research:
sediment contamination is widely used as a proxy for assessing human-induced impacts in aquatic ecosystems.
Numerous studies have demonstrated that chemical contaminants—especially trace metals and persistent organic pollutants—
accumulate in sediments and can adversely affect benthic organisms and overall ecosystem health~\cite{ChiaiaHernandez2022}.
These impacts disrupt ecosystem structure and function, causing shifts in species composition, food web dynamics, and
nutrient cycling. Such synchronous ecological changes provide the foundation for regression-based analysis 
between contamination levels and indicators of ecosystem condition. 
Among these indicators, the taxonomic composition of benthic macroinvertebrates is frequently used 
due to its sensitivity to sediment conditions and its practicality and cost-effectiveness 
compared to other biological measures.

Building on this relationship, one promising approach is to develop a model that links sediment contamination 
levels to shifts in benthic macroinvertebrate taxonomic composition. A simplified method involves constructing a 
composite index based on selected taxa and their relative abundances, which enhances communicability with stakeholders
and simplifies interpretation. However, this index-based method inevitably sacrifices information compared to analyses
using the full taxonomic dataset, and the selection of taxa and the index construction process may introduce
subjectivity and limit generalizability\cite{Desrosiers2020}. 
\textcolor{blue}{On top of it, how to scale the index and measure the distance between indices is a
crucial work to reveal the meaning of changes and compare shifts in the index values, which originally 
is determined by the raw benthic macroinvertebrate data.}
It is also important to note that benthic macroinvertebrates respond not
only to anthropogenic chemical contaminants, but also to natural environmental variability such as sediment texture, 
organic content, and temperature. These complexities raise concerns about the reliability of such indices in attributing
observed community shifts specifically to human-induced stress, thereby challenging the validity of stressor–index models that exclude natural variation.

To address this issue, it is necessary to consider environmental attributes in model development. 
Environmental variables such as sediment characteristics, water chemistry, and hydromorphology play key roles
in structuring benthic communities. 
Therefore, the ability to distinguish natural variation from human-induced
impacts becomes a critical question. 
Including environmental covariates in regression models can analytically partition
their influence from that of anthropogenic stressors through multivariate analysis.
\textcolor{blue}{(Do we need env-variables in integrity assessment? if no, remove this mention!)}
However, in the context of ecological
integrity assessment or stress evaluation, this separation is often more complex due to the partial confounding
of natural and human-induced factors. While the confounding effects of environmental variables exists in both
modeling and assessment contexts, the methods required to account for them may differ substantially.

Additionally, the existence of unmeasured or unmeasurable factors—and their complex interactions 
with observed model inputs—may lead to non-linear relationships between variables, 
highlighting the importance of threshold detection. 
This detection introduces a crucial modeling component: identifying points at which small 
changes in environmental stressors lead to abrupt ecological responses. 
In the context of this study, such thresholds may reflect tipping points in sediment contamination 
levels, beyond which benthic community structures shift significantly. 
Detecting these thresholds enables more targeted and efficient bioassessment strategies, 
and may help guide the development of environmental quality criteria or inform restoration priorities.

\medskip

To apply these concepts effectively, \textbf{this program will narrow its focus to the Huron-Erie corridor}, 
a critical aquatic link connecting Lake Huron and Lake Erie. This corridor forms a 
hydrologically synchronous water system, characterized by faster flow 
caused by \textit{channel constriction}—a well-known concept in fluid dynamics. 
These physical characteristics contribute to unique environmental conditions in the corridor 
and increase the complexity of sediment assessment, making traditional assessment approaches 
less suitable for this setting.

\medskip

\textbf{Our goal is to develop a more economical and efficient method to assess sediment contamination levels} 
specifically tailored to the Huron-Erie corridor, with the potential for broader application 
to nearby aquatic ecosystems. The foundational idea is inspired by the work of Jian 
(Zhang 2008)~\cite{Zhang2008}, who investigated the composition of zoobenthic 
communities to infer contamination in this region.

\textbf{Building on Jian's general framework}, we aim to both enhance the original methodology and 
incorporate recent advancements—enabled by improved computational resources and 
emerging context-specific analytical techniques. Through this effort, we seek to create a 
more adaptive, data-driven approach to infer aquatic condition from zoobenthic measurements in complex freshwater systems.