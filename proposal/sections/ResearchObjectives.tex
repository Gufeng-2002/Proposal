\section{Research Objectives}

The goal of this study is to build a zoobenthic community indicator(ZCI) of sediment contamination 
in a large aquatic ecosystem(e.g., SCDRS or Detroit River) by means of multivariate statistical analysis.
This work can be divided into five specific objectives with respects to aquatic ecology principles and
statistical methodologies:

\begin{enumerate}
    \item Build quantitative composite measures of sediment contamination levels and zoobenthic community composition.
        \item Control for natural variability in community composition and isolate anthropogenic effects on community composition.
        \item Incorporate spatial features to account for potential spatial heterogeneity in community composition.
        \item Model the relationship between the sediment contamination levels and zoobenthic community composition
        with piecewise quantile regression and extend it to a bioindicator (ZCI) of sediment contamination level.
        \item Evaluate the indicator power and robustness of the developed ZCI with respect
        to sample size and other relevant global factors that can influence the estimates.
\end{enumerate}

These listed objectives are made in a tentatively sequential order. Some objectives have been well-supported by
existing researches while others are more exploratory. Objectives \textcolor{blue}{1, 2} and \textcolor{blue}{4},
form a good benchmark of building the zoobenthic community indicator and are supported by a well-established
\textbf{Multivariate Community Indicator Framework}
\footnote{Its details are provided in following content and in the Methodology section}
\cite{Brazner2007, Kovalenko2014, Host2019, Ciborowski2005ZoobenthicIndicators,Zhang2008}
amenable to application of a piecewise quantile regression model.
Objectives \textcolor{blue}{3} and \textcolor{blue}{5} require exploratory investigations, 
and will serve to enhance and increase understanding of the indicator's power and sensitivities.

The following subsections disccussed the proposed general approach, highlighting the focuses where possible contributions
can be made.

\subsection{Multivariate community indicator framework}

Multivariate community indicator framework is a comprehensive approach for achieving objectives \textcolor{blue}{1, 2} and \textcolor{blue}{4}, 
we will build it with possible improvements.
Three data matrices (Zhang 2008 \cite{Zhang2008}; Farara and Burt 1993 \cite{Farara1993}; Wood 2004 \cite{Wood2004}) will be used for it:
the taxa matrix (biological response), the environmental covariate matrix, and the stressor (chemical) matrix.
Furthermore, quantitative indices that summarize sediment contamination and zoobenthic community composition
will be designed and applied meanwhile controlling the natural variability. 
Its basic steps are summarized as follows:

\begin{enumerate}
    \item \textbf{Measure sediment contamination and select reference sites}
    
    \emph{Motivation: establish a defensible composite measure that reflects a contamination gradient and identify 
    least disturbed sites that serve as reference locations.}

    Perform Principal Component Analysis (PCA) on elements of the stressor matrix;
    filter and interpret qualified pollutant Principal Components (PCs) from which to form a
    composite contamination score. Designate sites with the scores indicative of minimally-disturbed
    levels as reference sites from which to identify their putative "reference condition" assemblages.

    \item \textbf{Predict 'reference condition' community composition across environmental gradients} 
    
    \emph{Motivation: predict the variation in community composition as controlled by environmental covariates
    to disentangle the pollution-driven deviation in the community composition.}

    Perform hierarchical agglomerative cluster analysis of the biota of reference-site samples to identify
    distinct groups of taxa; train a discriminant model to predict the biological assemblages present at 
    a location on the basis of the location’s environmental conditions. 
    Thus, sites having the same biological assemblage are considered environmentally homogeneous.

    \item \textbf{Measure community composition (ZCI) for environmentally homogeneous sites}

    \emph{Motivation: construct a numerical index for each biological assemblage to summarize
    community composition for sites with similar environmental conditions.}

    Within each predicted cluster, apply ordination (e.g., PCA/NMDS) to summarize taxa information;
    derive the measurement results as ZCI with defined scoring rules.
    Sites deviating from the reference site centroid are considered disturbed by pollution,
    they compromise the disturbance-relevant ZCI results.

    \item \textbf{Quantify the relationship between contamination level and ZCI}
    
    \emph{Motivation: quantify how community composition responds to contamination levels}

    Fit regression models of ZCI deviation versus contamination score to capture deterministic relationships 
        and stochastic variation that shapes the conditional distributional response.
        Determining a suitable relationship pattern, e.g., smooth linear or breakpoint-based piecewise,
        is the key to the first step.
        If the latter exists, the following tasks will be to identify (1) the level of contamination at which the breakpoint(s) occur and
        (2) the benchmark(s) of ZCI representative of these crucial contamination levels.
\end{enumerate}



\subsection{Spatial heterogeneity test and incorporation}

Objective \textcolor{blue}{3} is to ensure the unmodelled spatial patterns do not confound the regression results.
Ecological data often exhibit spatial autocorrelation—nearby sites tend to have similar communities—so
failing to include spatial structure can lead to biased estimates and inflated error rates.
The principal coordinates of neighbour matrices (PCNM) method will be applied to transform spatial distances among sites into orthogonal spatial
predictors that can be incorporated into regression or canonical analyses \cite{Borcard2002PCNM, Dray2006SpatialEigenfunction, GriffithPeresNeto2006SpatialFiltering}.

PCNM eigenvectors from the Euclidean distance matrix will capture spatial structure at different scales \cite{Borcard2002PCNM, GriffithPeresNeto2006SpatialFiltering}.
For each environmental cluster of sample sites, we will test for spatial heterogeneity by regressing the ZCI 
derived for that group of sites against PCNM vectors 
to identify significant spatial predictors not explained by environmental variables.
Selected PCNM variables will be incorporated as covariates in quantile regression models to control spatial autocorrelation.
% Variance partitioning will help control the correlation among contamination levels across space,
% improving the accuracy in estimating the ZCI-contamination relationship.

\subsection{Piecewise quantile regression for breakpoint in quantile relationship}

Objective~\textcolor{blue}{4} is to use a piecewise quantile regression to model the relationship 
between sediment contamination levels and a ZCI,
which identifies breakpoints in the ZCI-contamination relationship within the multivariate community indicator framework. 
Piecewise quantile regression (PQRM) is chosen for its ability to estimate conditional quantiles 
of a response without assuming a specific parametric form \cite{Cade2003,Huang2017}, 
to model the entire conditional distribution rather than just the mean \cite{Cade2003},
and to estimate parameters that are robust to outliers or heteroscedastic variance \cite{Huang2017}.
In the ecological context, quantile regression can reveal abrupt changes in zoobenthic community 
composition associated with specific contamination levels at different quantiles 
(e.g., high quantile representing sensitive taxa)
and detect breakpoints that indicate ecological thresholds \cite{Jabed2020,Spake2022,Daily2012}.

It is informative to detect threshold(s) in contamination levels across which the slope of the ZCI–contamination relationship 
changes abruptly, indicating points where benthic communities change their way in responding to contamination. 
Higher quantiles may reveal steeper changes than median quantiles, highlighting vulnerable taxa. 
Comparing breakpoints across environmentally distinct clusters (if the comparison across environmental gradients is feasible)
will show whether thresholds vary among environmental gradients.


\subsection{Indicator power and robustness with respect to sample size}
The goal of Objective~\textcolor{blue}{5} is to evaluate how reliably
the ZCI detects sediment contamination gradients under varying sampling restrictions and conditions.
Two primary targets can be (1) exploring the minimum size of training data that supports the reliable estimation 
of key parameters (e.g., slopes, breakpoints)\cite{Spake2022} in the ZCI–contamination relationship; 
(2) how large a test set should be used to determine whether a new site falls above or below 
particular threshold(s) with acceptable probabilistic confidence when it truly does so.


Robustness analyses are essential because bioindicator performance can deteriorate when sample sizes are small, variance components are poorly estimated, or site selection produces hidden pseudoreplication \cite{Hurlbert1984Pseudo}. Power and precision directly influence management credibility; underpowered indicators risk Type II errors (failing to flag degraded conditions) while unstable estimates inflate Type I error rates in threshold detection \cite{Osenberg1994ImpactPower,Fairweather1991MonitoringPower}.

The analytical approach is tentatively to use bootstrap resampling and subsampling to evaluate ZCI robustness.
Bootstrap resampling may generate sampling distributions for key parameters (slopes, breakpoints, pseudo-$R^2$),
while subsampling could construct precision curves by repeatedly selecting subsets of sites across sample sizes.
Power analysis will potentially simulate datasets under null and alternative hypotheses to compute detection power
for contamination effects and threshold shifts. Spatial autocorrelation may need to adjust effective sample sizes,
and breakpoint reliability could be assessed through confidence interval width criteria or other suitable metrics.


Results will identify optimal sampling design (sites per cluster) and provide decision-support tables
linking confidence levels to required sampling effort. Higher quantiles may require larger samples due to greater dispersion,
while also elevating the potential concerns regarding spatial autocorrelation.





