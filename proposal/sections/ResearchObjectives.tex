\section{Research Objectives}

\subsection{General objectives}

The goal of this thesis is to enhance the analytical processes in Jian's work and 
to further design and implement the core idea of using zoobenthic measurements to 
infer the sediment contamination level of the Huron-Erie corridor.
Specifically, the general objectives can be summarized as follows:
\begin{enumerate}
    \item Assess anthropogenic adverse effects on aquatic sites using stressor measurements.
    \item Design and implement a zoobenthic community indicator to assess taxonomic composition structure.
    \item Explore the potential for using the zoobenthic community indicator 
    to infer sediment contamination level with control of environmental variables.
    Design an inference model if appropriate.
\end{enumerate}

\subsection{Overview of the sub-objectives with the proposed framework}

The above general goals will be supported and further detailed by the following specific objectives and corresponding analytical steps:
\begin{enumerate}
    \item \textbf{Sediment contamination assessment}
    
    Apply an assessment method to chemical element concentrations to evaluate pollutant levels, 
    interpret pollutant patterns among sites, and identify key contaminants.

    \item \textbf{Zoobenthic community structure discriminant model of environmental variables}

    Filter reference sites based on assessed stress levels, assuming that human impact is minimal or absent, 
    and that the community structure is primarily shaped by natural environmental variability.

    Explore the relationship between minimally impacted community structure and environmental variables, 
    and build a predictive model to infer the expected community structure under natural conditions (without human disturbance).
    
    Given the limited environmental variables and other potentially unmeasured or unmeasurable environmental factors, 
    it is nearly impossible to train a fully quantitative inference model from environmental variables to taxa.

    Therefore, constrained predicted values should be constructed and applied to avoid overfitting 
    and to extract limited yet informative inferences about community structure using available environmental variables.

    \textit{It requires the following additional steps:}:
    \begin{itemize}
        \item \textbf{Partition the reference sites into different groups} based on their taxa composition.
        \item Build a discriminant model \textbf{between the taxa composition groups and environmental variables}.
    \end{itemize}

    The resulting predicted group labels provide more reliable information, as the limited 
    environmental variables are only used to classify sites into taxa composition groups that were 
    predefined from the reference sites.

    \item \textbf{Partition all sites into comparable taxa composition groups}
    
    Apply the discriminant model to all sites, including degraded and intermediate sites, 
    to assign each site to one of the taxa composition groups. 
    The group positions are fixed based on reference site partitioning.

    \item \textbf{Measure the distance between each site and the reference and degraded endpoints within each taxa composition group}

    Scale the taxa composition structure and define a metric to measure the 
    distance between each site and the reference and degraded endpoints within each group.
    Given similar environmental conditions across the group, differences in taxa composition are assumed to be caused by human disturbance.

    \item \textbf{Quantitative regression to assess the relationship between the stressor level and the taxa composition structure}
    
    Variations in taxa composition from the reference endpoint should be explained by the 
    relative stress level of each site. A quantitative regression model can assess this relationship.
    Considering the large number of potential ecological influences, non-linear quantile regression is employed 
    to explore patterns beyond traditional mean regression.

\end{enumerate}

\subsection{Potential Impacts and Further Questions}

Through achieving these objectives, there is a potential to significantly enhance the understanding of
the sediment contamination within the \textcolor{blue}{Huron-Erie corridor} and to develop 
a more robust zoobenthic community indicator that can be used to infer sediment contamination levels.

\textbf{Several benthic integrity assessment relevant questions arise and might be answered from the implementation of these objectives},
tentatively listed as follows:
\begin{itemize}
    \item Is there a generally applicable sediment contamination assessment method that can be applied to different types of 
    environmental conditions, or does the method need to be tailored to specific conditions?

    \item To what extent the clustering-discriminant model
    \footnote{The clustering and discriminant methods are bundled together to form a single step in the framework. 
    Its purpose is to extract confined information(groups) from the selected sites, 
    fit a discriminant model of environmental variables
    for the groups, and then apply the model to all sites to assign them to the groups.}
    can effectively capture the natural variability in taxa composition,
    and how the clustering parts, which are applied on different \textbf{partitions} in contamination assessment results,
    can shape the discriminant model?

    \item How to construct the measurement of zoobenthic community structure according to the contamination assessment results 
    to support the inference model? To what extent should the two measurements be correlated but meanwhile keeping partial independence 
    to ensure their respective interpretability?

    \item How to link the measurement of zoobenthic community structure to probability measurement of sediment contamination level
     - degraded, especially in areas with intermediate anthropogenic influence.
\end{itemize}





