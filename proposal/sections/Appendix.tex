\section{Appendix}

\subsection{Tables}

\begin{table}[htbp]
\centering
\caption{Environmental Variables and Their Explanations}
\label{tab:env_variables}
\renewcommand{\arraystretch}{1.3}
\begin{tabular}{|>{\centering\arraybackslash}m{6cm}|>{\centering\arraybackslash}m{8.5cm}|}
\hline
\textbf{Variable Name} & \textbf{Explanation} \\
\hline
Site\_ID & Unique identifier for each sampling site \\
Lake\_or\_River & Indicates whether the site is in a lake or river \\
Latitude & Geographic latitude coordinate \\
Longitude & Geographic longitude coordinate \\
Total\_Organic\_Carbon\_LOI\_percent & Total organic carbon content (loss on ignition, as \%) \\
Water\_Depth\_m & Water depth at the sampling location (meters) \\
Water\_Temperature\_C & Water temperature in degrees Celsius \\
Dissolved\_Oxygen\_Concentration\_mgL & Dissolved oxygen concentration in milligrams per liter \\
Median\_Particle\_Size\_Phi & Median particle size of sediment (Phi scale) \\
\hline
\end{tabular}
\end{table}

\begin{table}[htbp]
\centering
\caption{Taxonomic Variables and Their Explanations}
\label{tab:taxonomic_variables}
\renewcommand{\arraystretch}{1.3}
\begin{tabular}{|>{\centering\arraybackslash}m{4.5cm}|>{\centering\arraybackslash}m{7.5cm}|}
\hline
\textbf{Taxonomic Group} & \textbf{Explanation} \\
\hline
Oligochaeta & Aquatic segmented worms \\
Nematoda & Roundworms \\
Chironomidae & Non-biting midges (larvae) \\
Ceratopogonidae & Biting midges \\
Hexagenia & Mayfly genus (larvae) \\
Caenis & Mayfly genus (larvae) \\
Hydropsychidae & Net-spinning caddisflies \\
Other Trichoptera & Other caddisfly families \\
Amphipoda & Small crustaceans (e.g., scuds) \\
Dreissena & Zebra/quagga mussels \\
Acari & Aquatic mites \\
Hydrozoa & Small predatory animals (hydroids) \\
Hirudinea & Leeches \\
Turbellaria & Flatworms \\
Gastropoda & Snails and slugs \\
Sphaeriidae & Fingernail clams \\
\hline
\end{tabular}
\end{table}

\begin{table}[htbp]
\centering
\caption{Stressors and Their Explanations}
\label{tab:stressors}
\renewcommand{\arraystretch}{1.3}
\begin{tabular}{|>{\centering\arraybackslash}m{3.5cm}|>{\centering\arraybackslash}m{8.5cm}|}
\hline
\textbf{Chemical/Contaminant} & \textbf{Explanation} \\
\hline
Al & Aluminum (trace metal) \\
As & Arsenic (toxic element) \\
Bi & Bismuth (trace element) \\
Ca & Calcium (major element, hardness) \\
Cd & Cadmium (toxic metal) \\
Co & Cobalt (trace element) \\
Cr & Chromium (trace metal) \\
Cu & Copper (trace metal, micronutrient) \\
Fe & Iron (major element, micronutrient) \\
Hg & Mercury (highly toxic metal) \\
K & Potassium (major element) \\
Mg & Magnesium (major element, hardness) \\
Mn & Manganese (trace element) \\
Na & Sodium (major element) \\
Ni & Nickel (trace metal) \\
Pb & Lead (toxic metal) \\
Sb & Antimony (trace element) \\
V & Vanadium (trace element) \\
Zn & Zinc (trace metal, micronutrient) \\
\%OC & Percent organic carbon \\
1245-TCB & 1,2,4,5-Tetrachlorobenzene (organic pollutant) \\
1234-TCB & 1,2,3,4-Tetrachlorobenzene (organic pollutant) \\
QCB & Quintachlorobenzene (organic pollutant) \\
HCB & Hexachlorobenzene (organic pollutant) \\
OCS & Octachlorostyrene (organic pollutant) \\
p,p'-DDE & DDT breakdown product \\
p,p'-DDD & DDT breakdown product \\
mirex & Organochlorine insecticide \\
Heptachlor Epoxide & Organochlorine pesticide breakdown product \\
total PCB & Total polychlorinated biphenyls \\
\hline
\end{tabular}
\end{table}

\begin{table}[htbp]
\centering
\caption{Explanation, Habitat, and Survival Rate in Fast/Slow Water for Each Taxon}
\begin{tabular}{|c|c|c|c|}
\hline
\textbf{Taxa} & \textbf{Explanation} & \textbf{Habitat Type} & \textbf{Survival Rate (Fast/Slow Water)} \\
\hline
Oligochaeta & Aquatic segmented worms & Both (lentic/lotic) & Moderate/High \\
Nematoda & Roundworms & Both (lentic/lotic) & Moderate/High \\
Chironomidae & Non-biting midge larvae & Both (lentic/lotic) & Moderate/High \\
Ceratopogonidae & Biting midge larvae & Both (lentic/lotic) & Moderate/Moderate \\
Hexagenia & Burrowing mayflies & Lentic (lakes/ponds) & Low/High \\
Caenis & Small mayflies & Both (lentic/lotic) & Low/Moderate \\
Hydropsychidae & Net-spinning caddisflies & Lotic (streams/rivers) & High/Low \\
Other Trichoptera & Other caddisflies & Both (lentic/lotic) & Moderate/Moderate \\
Amphipoda & Small crustaceans & Both (lentic/lotic) & Moderate/High \\
Dreissena & Zebra/quagga mussels & Lentic, large rivers & Low/High \\
Acari & Aquatic mites & Both (lentic/lotic) & Moderate/High \\
Hydrozoa & Predatory invertebrates & Lentic (lakes/ponds) & Low/High \\
Hirudinea & Leeches & Both (lentic/lotic) & Low/High \\
Turbellaria & Flatworms & Both (lentic/lotic) & Low/High \\
Gastropoda & Aquatic snails & Both (lentic/lotic) & Low/High \\
Sphaeriidae & Fingernail clams & Both (lentic/lotic) & Low/High \\
\hline
\end{tabular}
\end{table}


% statistical description for the chemical data within reference, intermediate, and degraded sites based on PCA assessment
\begin{table}[ht]
\centering
\caption{Chemical Descriptive Statistics by Site Label}
\label{tab:chem_desc}
\begin{tabular}{>{\centering\arraybackslash}m{2.5cm} >{\centering\arraybackslash}m{1.5cm} >{\centering\arraybackslash}m{2cm} >{\centering\arraybackslash}m{2cm} >{\centering\arraybackslash}m{2cm}}
\toprule
 & \textbf{SumReal} & \textbf{degraded} & \textbf{intermediate} & \textbf{reference} \\
\midrule
\multirow[t]{2}{*}{Al} & mean & 4276.423 & 6380.140 & 4319.381 \\
 & std & 2888.769 & 5523.949 & 1767.861 \\
\cline{1-5}
\multirow[t]{2}{*}{As} & mean & 2.186 & 1.777 & 2.232 \\
 & std & 1.602 & 1.290 & 1.041 \\
\cline{1-5}
\multirow[t]{2}{*}{Bi} & mean & 17.085 & 17.505 & 17.622 \\
 & std & 10.352 & 10.273 & 9.722 \\
\cline{1-5}
\multirow[t]{2}{*}{Ca} & mean & 28180.500 & 33518.930 & 28480.714 \\
 & std & 14031.433 & 11400.266 & 11870.107 \\
\cline{1-5}
\multirow[t]{2}{*}{Cd} & mean & 0.535 & 0.351 & 0.271 \\
 & std & 0.649 & 0.202 & 0.233 \\
\cline{1-5}
\multirow[t]{2}{*}{Co} & mean & 4.049 & 4.497 & 3.984 \\
 & std & 1.733 & 2.209 & 1.118 \\
\cline{1-5}
\multirow[t]{2}{*}{Cr} & mean & 13.254 & 12.830 & 9.007 \\
 & std & 16.373 & 11.835 & 2.937 \\
\cline{1-5}
\multirow[t]{2}{*}{Cu} & mean & 16.958 & 18.082 & 12.946 \\
 & std & 22.388 & 29.120 & 9.003 \\
\cline{1-5}
\multirow[t]{2}{*}{Fe} & mean & 9495.000 & 11246.789 & 9650.905 \\
 & std & 5392.824 & 6804.654 & 3856.739 \\
\cline{1-5}
\multirow[t]{2}{*}{Hg} & mean & 0.474 & 0.324 & 0.196 \\
 & std & 1.230 & 0.420 & 0.365 \\
\cline{1-5}
\multirow[t]{2}{*}{K} & mean & 818.927 & 1285.558 & 845.657 \\
 & std & 638.053 & 1092.550 & 411.332 \\
\cline{1-5}
\multirow[t]{2}{*}{Mg} & mean & 12849.500 & 15204.175 & 12269.143 \\
 & std & 6104.202 & 5764.037 & 5281.794 \\
\cline{1-5}
\multirow[t]{2}{*}{Mn} & mean & 161.228 & 188.900 & 161.905 \\
 & std & 76.973 & 86.663 & 57.883 \\
\cline{1-5}
\multirow[t]{2}{*}{Na} & mean & 118.998 & 134.042 & 123.611 \\
 & std & 49.081 & 43.693 & 41.021 \\
\cline{1-5}
\multirow[t]{2}{*}{Ni} & mean & 11.225 & 12.399 & 9.136 \\
 & std & 8.851 & 8.424 & 3.542 \\
\cline{1-5}
\multirow[t]{2}{*}{Pb} & mean & 12.515 & 8.774 & 8.573 \\
 & std & 32.312 & 22.204 & 18.750 \\
\cline{1-5}
\multirow[t]{2}{*}{Sb} & mean & 17.262 & 16.765 & 18.001 \\
 & std & 11.879 & 13.115 & 13.743 \\
\cline{1-5}
\multirow[t]{2}{*}{V} & mean & 15.274 & 18.353 & 15.183 \\
 & std & 7.012 & 9.560 & 4.408 \\
\cline{1-5}
\multirow[t]{2}{*}{Zn} & mean & 52.732 & 46.181 & 35.677 \\
 & std & 48.896 & 44.586 & 17.938 \\
\cline{1-5}
\multirow[t]{2}{*}{\%OC} & mean & 2.110 & 2.405 & 1.779 \\
 & std & 1.599 & 1.458 & 0.682 \\
\cline{1-5}
\multirow[t]{2}{*}{1245-TCB} & mean & 0.906 & 1.201 & 0.555 \\
 & std & 2.321 & 2.143 & 1.035 \\
\cline{1-5}
\multirow[t]{2}{*}{1234-TCB} & mean & 0.252 & 0.234 & 0.253 \\
 & std & 0.257 & 0.240 & 0.332 \\
\cline{1-5}
\multirow[t]{2}{*}{QCB} & mean & 0.729 & 1.255 & 0.636 \\
 & std & 1.015 & 3.055 & 0.871 \\
\cline{1-5}
\multirow[t]{2}{*}{HCB} & mean & 2.759 & 17.713 & 2.904 \\
 & std & 4.291 & 83.487 & 6.011 \\
\cline{1-5}
\multirow[t]{2}{*}{OCS} & mean & 1.213 & 1.502 & 0.721 \\
 & std & 3.395 & 3.606 & 1.874 \\
\cline{1-5}
\multirow[t]{2}{*}{p,p'-DDE} & mean & 0.679 & 0.485 & 0.324 \\
 & std & 1.255 & 0.930 & 0.328 \\
\cline{1-5}
\multirow[t]{2}{*}{p,p'-DDD} & mean & 3.879 & 0.772 & 0.862 \\
 & std & 14.634 & 1.039 & 0.923 \\
\cline{1-5}
\multirow[t]{2}{*}{mirex} & mean & 0.253 & 0.212 & 0.134 \\
 & std & 0.682 & 0.332 & 0.242 \\
\cline{1-5}
\multirow[t]{2}{*}{Heptachlor Epoxide} & mean & 0.098 & 0.051 & 0.071 \\
 & std & 0.235 & 0.250 & 0.211 \\
\cline{1-5}
\multirow[t]{2}{*}{total PCB} & mean & 15.137 & 10.705 & 7.715 \\
 & std & 32.189 & 36.285 & 16.795 \\
\cline{1-5}
\bottomrule
\end{tabular}
\end{table}

\subsection{Index-based methods for quantitative stress metrics}
\begin{tcolorbox}[colback=yellow!10!white,
                                        colframe=blue!80!black,
                                        title = Introduce the index-based methods for quantitative stress metrics,
                                        fonttitle=\bfseries] 

PCA is mainly for exploratory work, like finding patterns, associations in the data.
It may be used to score the sediment contamination level, but only comparing relative abundance 
of stressor elements among the sites, lacking a little standards to accurately assess the contamination level.

Index-based methods can be used to quantitatively assess the sediment contamination level.
Such method needs more empirical and theoretical support than PCA, but it offers a more standardized way
to assess the contamination level, which is theoretically more accurate.

\textbf{Combining the PCA and index-based methods} is a good idea to explore the sediment contamination level.
I am thinking about what do they mean in contamination assessment, and how to combine them with
a reasonable logical flow.

\end{tcolorbox}

\subsection{Principal Component Analysis based methods to explore contaminant association and patterns }

\textcolor{blue}{Discuss the reason why and how PCA can be used to assess the sediment contamination level and reflect the 
aquatic ecological integrity.}

In SVD, a data matrix $X$ is decomposed as $X = U \Sigma V^T$, where $U$ and $V$ are orthogonal matrices and
 $\Sigma$ is a diagonal matrix of singular values.
For PCA, the principal components correspond to
the directions of maximum variance, which are given by the right singular vectors in $V$.
By incorporating weights into the data matrix, Weighted PCA modifies the SVD process 
to emphasize certain observations(row or column), allowing for more flexible dimensionality reduction 
tailored to the importance of each data point.

Given a data matrix $X \in \mathbb{R}^{m \times n}$, the SVD decomposes $X$ as $X = U \Sigma V^T$, where:
\begin{itemize}
    \item $U \in \mathbb{R}^{m \times m}$ contains the left singular vectors (columns),
    \item $\Sigma \in \mathbb{R}^{m \times n}$ is a diagonal matrix of singular values,
    \item $V \in \mathbb{R}^{n \times n}$ contains the right singular vectors (columns).
\end{itemize}

To solve for $U$ and $V$:
\begin{enumerate}
    \item Compute $X^T X$ and find its eigenvectors and eigenvalues. The eigenvectors form the columns of $V$.
    \item It can be proven that \(\mu_i^T = \frac{X v_i}{\sigma_i}\) is the $i$-th column of $U$, where $\sigma_i$ is the $i$-th singular value(square root of eigenvalue \(\lambda_i\)).
    \item The singular values in $\Sigma$ are the square roots of the nonzero eigenvalues from either $X X^T$ or $X^T X$.
\end{enumerate}

Mathematically, given a matrix $X$ of shape $(m, n)$, the SVD can be expressed as: 
\[
(X^T X) V = V \Lambda, \quad \Lambda = \text{diag}(\lambda_1, \lambda_2, \ldots, \lambda_n)
\]
According to spectral theorem, the eigenvalues $\lambda_i$ are non-negative and can be ordered as $\lambda_1 \geq \lambda_2 \geq \ldots \geq \lambda_n \geq 0$.
\(V\) is an orthogonal matrix, meaning \(V^T V = I\), where \(I\) is the identity matrix.
It gives:
\[
(X^T X) = V \Lambda V^T
\]
To a centered sample matrix of size \(m\) with \(n\) features, its covariance matrix is \(\frac{1}{m-1} (X^T X)\).
Via the eigenvalue decomposition, the variation in it can be expressed by the paris of eigenvalues and eigenvectors,
which are a series of rank-one matrices that carry different and independent information:
\[
\frac{1}{m-1} (X^T X) = \frac{1}{m-1} \sum_{i=1}^{n} \lambda_i v_i v_i^T
\]

Therefore, when columns in \(X\) represent different features, the columns \(v_i\) of \(V\) are the principal components 
that have its values as the linear combinations of the original features, and the scaled eigenvalues \(\frac{\lambda_i}{m-1}\)
as the amount of variation explained by each principal component.

Weighted PCA leverages the Singular Value Decomposition (SVD)
to extract principal components from weighted data.

\textcolor{blue}{Preparing to explain why this ordinal decomposition is not good for our case, and how we can 
improve it by considering the weights for different chemical elements and filtering out the from the PCs}

\subsection{Hierarchical Clusering analysis for Zoobenthic Communicty Indicator Construction}

Application of Hierarchical Clustering in Stressor-Community Analysis
Overview
Hierarchical clustering offers a data-driven approach to group sampling sites based on environmental or 
community-level similarities. Unlike partitioning methods, hierarchical clustering builds a nested tree
(dendrogram) that captures the progressive grouping of observations based on their dissimilarities.
In the context of my program, hierarchical clustering can be leveraged to define natural environmental 
classes prior to modeling the biological response to stressors.

Let each observation \( x_i \in \mathbb{R}^p \) denote a site with p-dimensional attributes (e.g., environmental variables). 
The dissimilarity between two observations \( x_i \) and \( x_{i'} \) is measured by a distance function,
often chosen as the squared Euclidean distance:
\[ d(x_i, x_{i'}) = \sum_{j=1}^p (x_{ij} - x_{i'j})^2 \]

Based on the distance metric, we can define the similarity measure on variable and observation levels as the following:
\[
\begin{aligned}
    &\text{Variable level:} \quad d_j(x_{ij}, x_{i'j}) = (x_{ij} - x_{i'j})^2, \quad j = 1, \ldots, p \\
    &\text{Observation level:} \quad d(x_i, x_{i'}) = \sum_{j=1}^{p} (x_{ij} - x_{i'j})^2 \\
\end{aligned}
\]

To cluster level \( D_C \), we can define other alternatives to the dissimilarity measures, commonly including:

\[ D_{SL}(G, H) = \min_{i \in G,\; i' \in H} d(x_i, x_{i'}) \quad \text{(Single Linkage)} \]
\[ D_{CL}(G, H) = \max_{i \in G,\; i' \in H} d(x_i, x_{i'}) \quad \text{(Complete Linkage)} \]
\[ D_{GA}(G, H) = \frac{1}{|G||H|} \sum_{i \in G} \sum_{i' \in H} d(x_i, x_{i'}) \quad \text{(Average Linkage)} \] 

\begin{itemize}
    \item \textbf{Use in the Program:}
    \begin{itemize}
        \item Hierarchical clustering is used to categorize sampling sites into clusters with similar environmental conditions before building predictive models linking taxonomic composition to stressor levels.
        \item This enables a two-stage analysis:
        \begin{itemize}
            \item Use environmental variables to form environmental clusters (reference site classification).
            \item Model stressor-community relationships within or across these clusters to control for natural variation.
        \end{itemize}
    \end{itemize}
    \item \textbf{Advantages:}
    \begin{itemize}
        \item Does not require pre-specification of the number of clusters.
        \item Produces a dendrogram showing how clusters are formed step-by-step.
        \item Enables ecological interpretation of clusters through tree visualization.
        \item Allows separation of natural variation from anthropogenic stress impacts.
    \end{itemize}
    \item \textbf{Model-Based Interpretation:}
    \begin{itemize}
        \item Assume that each cluster \( k \) corresponds to a latent distribution \( p_k(x) \), with the overall mixture model:
        \[
        p(x) = \sum_{k=1}^K \pi_k p_k(x)
        \]
        \item Each observation is generated as \( x \sim p_k(x) \) conditional on cluster membership \( k \), providing a statistical grounding for hierarchical clustering in unsupervised structure discovery.
    \end{itemize}
\end{itemize}


\textcolor{blue}{waiting to add more specific details after applying the clustering on the data and getting
some preliminary results.}

% \textcolor{blue}{Discuss why Clustering analysis is needed and how it was conducted
% on the taxonomic composition data to group them into clusters of different environmental conditions.}

% It needs to explain why directly clustering on the taxonomic composition can group them 
% into clusters of different environmental conditions instead of contamination levels.
% Because the assumption is: taxonomic compostion reponds to both the environmental conditions and 
% the contamination levels. If we think the clustering results are domainated by the environmental 
% conditions, that implies the influence by environment is stronger than by the contamination levels.


% \textcolor{blue}{These sites within each clusters should stay in specific environmental conditions.}

% Hierarchical clustering is a method of cluster analysis that seeks to build a hierarchy of clusters,
% it supports users to choose the number of clusters by cutting the dendrogram at a desired level, and 
% the testing on how well the clustering results represent the data structure is also available.




\subsection{Piecewise Quantile Regression for Threshold Determination}

% \textcolor{blue}{Discuss why PQRM is better than common linear regression in detecting 
% potential thresholds}

Let $m_\tau(x; \boldsymbol{\beta}_\tau, \boldsymbol{\alpha}_\tau)$ be the $\tau$th quantile of the conditional distribution 
of the ecological response given environmental condition. Then we define:

\[
y_\tau = m_\tau(x; \boldsymbol{\beta}_\tau, \boldsymbol{\alpha}_\tau) + \varepsilon_\tau.
\]

The form is similar to the linear regression model, but there are no restrictions on the distribution of error term $\varepsilon_\tau$, 
which means the error terms can be heteroscedastic(non-constant variance).


Then the \textbf{PQRM} with two breakpoints is defined as:

\[
m_\tau(x_i; \boldsymbol{\beta}_\tau, \boldsymbol{\alpha}_\tau) =
\begin{cases}
\beta_{0\tau} + \beta_{1\tau} x_i & \text{for } x_i \leq \alpha_{1\tau} \\
\beta_{0\tau} + \beta_{1\tau} x_i + \beta_{2\tau}(x_i - \alpha_{1\tau}) & \text{for } \alpha_{1\tau} < x_i \leq \alpha_{2\tau} \\
\beta_{0\tau} + \beta_{1\tau} x_i + \beta_{2\tau}(x_i - \alpha_{1\tau}) + \beta_{3\tau}(x_i - \alpha_{2\tau}) & \text{for } x_i > \alpha_{2\tau}
\end{cases}
\tag{3}
\]

The \(\tau\) subscript indicates the quantile level.
Breakpoints are spotted in the predictor space, which seems uncorrelated with the response variable and its quantile,
but they are not fixed and should be estimated under different quantile levels so that
the piecewise model provides a better performance on the data with such quantile level.
That is, the breakpoints are also functions of the quantile level \(\tau\).
\[
Y_{\tau} = X(\alpha_{_({\tau}; 1)}, \alpha_{_({\tau}; 2)}) \cdot \boldsymbol{\beta_{\tau}} + \boldsymbol{\varepsilon_{\tau}}
\]

Because the model are not predicting the means but the quantiles of $y$'s at given $x$'s,
the measurement for the model performance should no longer be the mean squared error(MSE),
\footnote{If a model regresses the mean of $Y$, it should outperform other models in the measurement of MSE, assuming the assumptions are satisfied.}
the check function is the loss function, which is defined as:
\[
\rho_\tau(u) =
\begin{cases}
\tau u & \text{if } u \geq 0 \\
(\tau - 1) u & \text{if } u < 0
\end{cases}
\quad \text{where } u = y - \hat{y}
\]

The loss function is a random variable that derivatives from $Y$, and it also can be viewed as a function of 
the prediction $\hat{y_{\tau}}$, which is not a RV.
A specific quantile can be found by minimizing the expected loss function - $E({\rho_{\tau}(Y - \hat y_{\tau})})$ with respect to $\hat y_{\tau}$ across
the observations:
\[
    {\displaystyle q_{Y}(\tau )={\underset {\hat y_{\tau}}{\mbox{arg min}}}E(\rho _{\tau }(Y-\hat y_{\tau}))={\underset {\hat y_{\tau}}{\mbox{arg min}}}{\biggl \{}(\tau -1)\int _{-\infty }^{\hat y_{\tau}}(y-\hat y_{\tau})dF_{Y}(y)+\tau \int _{\hat y_{\tau}}^{\infty }(y-\hat y_{\tau})dF_{Y}(y){\biggr \}}} 
\]

The expectation is no more a random variable, but a function of $(\hat y_{\tau}, y, F_Y(y))$.
By computing the derivative of it with respect to $\hat y_{\tau}$, the equation is determined by $(\hat y_{\tau}, F_Y(y))$.
Settting it to zero and letting $q_{\tau}$ the solution of $\hat y_{\tau}$ to the equation, we have:

\[
    {\displaystyle 0=(1-\tau )\int _{-\infty }^{q_{\tau }}dF_{Y}(y)-\tau \int _{q_{\tau }}^{\infty }dF_{Y}(y).} 
\]

It gives:
\[
F_Y(q_{\tau}) = \tau
\]

Therefore, $\rho_\tau(u)$ is a valid loss function for inferring the quantile of $Y$.
However, when building quantile regression on $Y$ with $X$, we need to have enough observations on each condition of $X$.
Becasue there is no assumption that residuals are homoscedastic, the globally minimized loss function does not necessarily 
bring good predictions on all conditions, which is different to the linear regression model.

To a given condition $x$, the quantile of $\tilde y$ is defined as:
\[
    {\displaystyle q_{\tilde y|x}(\tau )={\underset {\hat y_{\tau}}{\mbox{arg min}}}E(\rho _{\tau }(\tilde y - \hat y_{\tau})|X)={\underset {\hat y_{\tau}}{\mbox{arg min}}}{\biggl \{}(\tau -1)\int _{-\infty }^{\hat y_{\tau}}(y - \hat y_{\tau})dF_{\tilde y|x}(y)+\tau \int _{\hat y_{\tau}}^{\infty }(y-\hat y_{\tau})dF_{\tilde y|x}(y){\biggr \}}} 
\]

The expected loss function that provides quantiles over all observations is:
\[
    E(\rho_{\tau}(\tilde y - \hat y_{\tau})) = \frac{1}{n} \sum_{i=1}^n \rho_\tau(\tilde y_i - x_i^\top \beta_{\tau})
\]

Optimizing it to minimum with respect to $\beta_{\tau}$ gives the optimal $\beta_{\tau}$
\footnote{There should be an assumption of linear relation on the amount changing between $x$ and $\tau$ quantile of $y$, and $\beta$ is the coefficient of that relation.}:
\[
\hat{\beta}\tau = {\underset {\beta_{\tau}}{\mbox{arg min}}} E(\rho_{\tau}(\tilde y - \hat y_{\tau}))
\]



When given two breakpoints, the $(\alpha_{(\tau; 1)}, \alpha_{(\tau; 2)})$ can be iteratively searched by 
Newton-Raphson method. Within each iteration, the loss function is minimized to find the optimal $\beta_{(\tau)}$ vector.

\textcolor{blue}{waiting to add more specific details after applying the clustering on the data and getting
some preliminary results.}


\subsection{Synthetic Data Generation}

Synthetic data generation refers to the process of artificially creating data that mimics 
the statistical properties and structure of real-world datasets. 
The core motivation is to generate data when real data are limited, imbalanced,
 private, or costly to obtain. Synthetic data are widely used in machine learning,
  data privacy preservation, and simulation-based research.

This approach can simulate various data types, 
including tabular (structured), image, text, and time series data. 
The generated data should reflect similar distributions, correlations,
 and interactions as the original data while maintaining flexibility and scalability.

Synthetic data generation can play a supportive role in this study by addressing several data-related challenges 
commonly encountered in ecological assessments:

\begin{itemize}
    \item \textbf{Enhancing model robustness}: By generating additional synthetic observations that mimic the real data structure, we can augment the existing data pool, which helps reduce model overfitting and improve generalization when predicting stressor impacts across unsampled sites.
    
    \item \textbf{Balancing site distribution across gradients}: Many ecological datasets are imbalanced with respect to environmental gradients or stressor intensity levels. Synthetic data can help balance the representation of different ecological zones or stressor conditions, ensuring the model learns from a more evenly distributed set of scenarios.
    
    \item \textbf{Supporting rare condition modeling}: Stressor levels or taxa responses in extreme or under-sampled regions (e.g., highly degraded or pristine sites) can be underrepresented. Synthetic data can simulate these rare cases to aid in threshold detection or to inform prediction in sparsely observed domains.
    
    \item \textbf{Facilitating model testing and validation}: Controlled synthetic datasets can be used to evaluate the behavior and sensitivity of the modeling framework under different stressor-environment-taxa scenarios, helping identify model biases or limits.
    
    \item \textbf{Exploring uncertainty and variability}: Synthetic data allow the introduction of controlled noise and variability, which is useful for evaluating how robust the model is to observational or measurement uncertainty.
\end{itemize}

A simple yet effective approach to synthetic data generation is random sampling (bootstrapping) from the observed dataset.
 Given a dataset \( \mathcal{D} = \{x_1, \dots, x_N\} \), where \( x_i \in \mathbb{R}^p \), 
 synthetic samples are generated by sampling with replacement:
\[
x_i^* \sim \hat{F}, \quad i = 1, \dots, N^*
\]
where \( \hat{F} \) is the empirical distribution of \( \mathcal{D} \), and \( N^* \) is
 the desired number of synthetic samples.

This method preserves the multivariate structure by sampling entire rows, maintaining correlation between features.
 It is non-parametric and computationally efficient.

\textbf{Use in this project:} Random sampling can augment small or imbalanced training sets, support bootstrapped 
threshold estimation, and improve model robustness across environmental gradients.

More common methods for synthetic data generation are summarized in Table \ref{tab:synthetic_data_methods}.

\begin{table}[!h]
\centering
\caption{Summary of common synthetic data generation methods and their applications.}
\label{tab:synthetic_data_methods}
\renewcommand{\arraystretch}{1.3}
\small
\begin{tabular}{|>{\centering\arraybackslash}p{3.5cm}|>{\centering\arraybackslash}p{7cm}|>{\centering\arraybackslash}p{4.2cm}|}
\hline
\textbf{Method} & \textbf{Description} & \textbf{Example Use Case} \\
\hline
\textbf{Random Sampling} & Generate synthetic data by sampling from known statistical distributions (e.g., normal, uniform, Poisson) & Simulating sensor readings or generating test input data \\
\hline
\textbf{SMOTE (Synthetic Minority Oversampling Technique)} & Generates new instances of the minority class by interpolating between existing instances & Addressing class imbalance in classification problems \\
\hline
\textbf{GANs (Generative Adversarial Networks)} & Uses a generator and a discriminator to iteratively create highly realistic synthetic data that mimics the real distribution & Synthetic images, medical data, or tabular data with complex dependencies \\
\hline
\textbf{VAEs (Variational Autoencoders)} & Learns a probabilistic latent space from real data and generates synthetic samples by decoding random samples from that space & Generating synthetic patient records or anomaly detection \\
\hline
\textbf{Agent-based Simulations} & Models behavior of individuals or agents and their interactions to generate data over time & Simulating traffic systems or disease spread \\
\hline
\textbf{Rule-based Generation} & Applies predefined rules and logical templates to construct synthetic data & Automated form testing, fake transactions for system validation \\
\hline
\end{tabular}
\end{table}

\textcolor{blue}{waiting to add more specific details after applying the clustering on the data and getting
some preliminary results.}
