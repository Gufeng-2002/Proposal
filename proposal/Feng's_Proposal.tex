\documentclass{article}
\usepackage{geometry}
\geometry{a4paper, 
margin=1in}
\usepackage{longtable}
\usepackage{amsmath}
\usepackage{amssymb}
\usepackage{booktabs}
\usepackage{xcolor}
\usepackage{hyperref}
\usepackage{graphicx} % Required for including images
\usepackage{subcaption}  % Required for subfigures
\usepackage{tcolorbox}  % For colored boxes
% \geometry{margin=1in}
\usepackage{booktabs}    % \toprule, \midrule, \bottomrule
\usepackage{caption}     % if you need \caption* or extra caption control
\usepackage{longtable}   % if your table might span multiple pages
\usepackage{array}       % for more flexible column specs (e.g. >{\ttfamily}l)
\usepackage{multirow}
\usepackage{tabularx}
\usepackage{array}
\usepackage{multirow}
\usepackage{booktabs}

% \title{Zoobenthic Community Indicators of Sediment Contamination}

% \author{Feng Gu}

\begin{document}
% \maketitle

\begin{titlepage}
\centering
\vspace*{2cm}

{\LARGE\bfseries Zoobenthic Community Indicators of Sediment Contamination\par}
\vspace{1.2cm}

{\Large\textbf{Research Proposal}\par}
\vspace{1.5cm}

{\large\textbf{Author:} Feng Gu\par}
\vspace{0.5cm}

{\large\textbf{Program and Institution:}\par}
\vspace*{.2cm}
{\large Master of Science (M.Sc.) in Data Science\\
\vspace*{.2cm}
Thompson Rivers University (TRU), Kamloops, British Columbia, Canada\par}
\vspace{1.5cm}

{\Large\textbf{Supervisory Committee:}\par}
\vspace*{.2cm}
{\large
Dr. Jabed H. Tomal (Supervisor) — Department of Mathematics and Statistics, TRU\\
\vspace*{.2cm}
Dr. Yue Zhang (Co-supervisor) — Department of Mathematics and Statistics, TRU\\
\vspace*{.2cm}
Prof. Jan Ciborowski (Committee Member) — Department of Biological Sciences, University of Calgary
\par}

\vfill
{\large October 2025\par}

\end{titlepage}


\clearpage

\begin{center}
    {\Large \textbf{Abstract}}
\end{center}

This proposal reviews the rationale and analytical framework of Zoobenthic Community Indicators (ZCI)
of Sediment Contamination in aquatic ecosystems. 
Building upon this foundation, the author outlines the research objectives and discusses
the key methodologies to be applied.
Preliminary results are produced based on the framework described in the \textit{Methodology} section,
demonstrating the feasibility of the proposed approach 
and supporting the achievability and significance of the research objectives.  

This proposal comprises eight main sections. 
The \textbf{Introduction} presents the research topic, describes the general relationships among its fundamental components and 
introduces the study system. 
The \textbf{Research Objectives} section defines the specific goals and principal questions to be addressed in the project. 
The \textbf{Data Description} outlines the dataset characteristics, including data types, sources, collection methods, and variable definitions. 
The \textbf{Literature Review} synthesizes relevant research on the close study objectives and their analytical approaches.
The \textbf{Methodology} section presents the overall analytical framework and discusses specific techniques, with several figures provided 
to illustrate the idealized workflow. 
The \textbf{Preliminary Results} section reports initial outcomes from implementing key methodological steps, demonstrating both 
the feasibility and scalability of the proposed framework. 
The \textbf{Practical Implementation Plan} describes the computational strategies supporting the research objectives, including 
the structure of the codebase and choice of programming languages. 
The \textbf{Supervisory Dissolution} section summarizes supervision conditions and acknowledgements, 
and the \textbf{Timeline} outlines a provisional schedule with major milestones for completion of each research phase.  


\clearpage
\tableofcontents

\clearpage
% Temporarily commented out due to missing tcolorbox package
% \newcommand{\noticingbox}[2]{
%     \begin{tcolorbox}[colback=yellow!10!white,
%                                         colframe=orange!90!black,
%                                         title = #2,
%                                         fonttitle=\bfseries]
%         #1
%     \end{tcolorbox}
% }

% Alternative simple implementation without tcolorbox
\newcommand{\noticingbox}[2]{
    \begin{center}
        \fbox{\begin{minipage}{0.8\textwidth}
            \textbf{#2}\\
            #1
        \end{minipage}}
    \end{center}
}

\newcommand{\refcolor}[1]{\textcolor{blue}{\ref{#1}}
} % Custom commands for convenience
\section{Introduction}

% \textcolor{blue}{(Background about the study area)}

The Great Lakes, which occupy 84\% 
of North America's surface fresh water and 21\% of the world's supply of surface fresh water
\cite{EPA_Greatlakes2024}, are one of the world's largest surface freshwater ecosystems. 
Many nutrients and contaminants in the Great Lakes are stored in the sediments, playing
a crucial role in supporting aquatic habitats and influencing water quality.

Contaminants in the sediments are, however, unavoidable.
Among the regions, heavily developed shorelines are found around Lake Erie
and the connecting channels such as the Detroit River,
where human activities have left pronounced environmental footprints~\cite{EPA_SOGL2007}. 
These impacts have raised public concerns about the potential ecological risks, 
and have prompted calls for stronger protective and restorative actions 
to safeguard these aquatic ecosystems.

In light of these challenges, scientific assessment of aquatic condition becomes fundamental 
to retaining the integrity of an ecosystem. 
One effective approach is \textbf{to assess the nature and extent of sediment contamination}, 
which directly reflects anthropogenic impacts and serves as a partial indicator of ecological integrity.

This rationale is grounded in a well-established body of research:
sediment contamination is widely used as a proxy for assessing human-induced impacts in aquatic ecosystems.
Numerous studies have demonstrated that chemical contaminants—especially trace metals and persistent organic pollutants—
accumulate in sediments and can adversely affect benthic organisms and overall ecosystem health~\cite{ChiaiaHernandez2022}.
These impacts disrupt ecosystem structure and function, causing shifts in species composition, food web dynamics, and
nutrient cycling. Such synchronous ecological changes provide the foundation for regression-based analysis 
between contamination levels and indicators of ecosystem condition. 
Among these indicators, the taxonomic composition of benthic macroinvertebrates is frequently used 
due to its sensitivity to sediment conditions and its practicality and cost-effectiveness 
compared to other biological measures.

Building on this relationship, one promising approach is to develop a model that links sediment contamination 
levels to shifts in benthic macroinvertebrate taxonomic composition. A simplified method involves constructing a 
composite index based on selected taxa and their relative abundances, which enhances communicability with stakeholders
and simplifies interpretation. However, this index-based method inevitably sacrifices information compared to analyses
using the full taxonomic dataset, and the selection of taxa and the index construction process may introduce
subjectivity and limit generalizability\cite{Desrosiers2020}. 
\textcolor{blue}{On top of it, how to scale the index and measure the distance between indices is a
crucial work to reveal the meaning of changes and compare shifts in the index values, which originally 
is determined by the raw benthic macroinvertebrate data.}
It is also important to note that benthic macroinvertebrates respond not
only to anthropogenic chemical contaminants, but also to natural environmental variability such as sediment texture, 
organic content, and temperature. These complexities raise concerns about the reliability of such indices in attributing
observed community shifts specifically to human-induced stress, thereby challenging the validity of stressor–index models that exclude natural variation.

To address this issue, it is necessary to consider environmental attributes in model development. 
Environmental variables such as sediment characteristics, water chemistry, and hydromorphology play key roles
in structuring benthic communities. 
Therefore, the ability to distinguish natural variation from human-induced
impacts becomes a critical question. 
Including environmental covariates in regression models can analytically partition
their influence from that of anthropogenic stressors through multivariate analysis.
\textcolor{blue}{(Do we need env-variables in integrity assessment? if no, remove this mention!)}
However, in the context of ecological
integrity assessment or stress evaluation, this separation is often more complex due to the partial confounding
of natural and human-induced factors. While the confounding effects of environmental variables exists in both
modeling and assessment contexts, the methods required to account for them may differ substantially.

Additionally, the existence of unmeasured or unmeasurable factors—and their complex interactions 
with observed model inputs—may lead to non-linear relationships between variables, 
highlighting the importance of threshold detection. 
This detection introduces a crucial modeling component: identifying points at which small 
changes in environmental stressors lead to abrupt ecological responses. 
In the context of this study, such thresholds may reflect tipping points in sediment contamination 
levels, beyond which benthic community structures shift significantly. 
Detecting these thresholds enables more targeted and efficient bioassessment strategies, 
and may help guide the development of environmental quality criteria or inform restoration priorities.

\medskip

To apply these concepts effectively, \textbf{this program will narrow its focus to the Huron-Erie corridor}, 
a critical aquatic link connecting Lake Huron and Lake Erie. This corridor forms a 
hydrologically synchronous water system, characterized by faster flow 
caused by \textit{channel constriction}—a well-known concept in fluid dynamics. 
These physical characteristics contribute to unique environmental conditions in the corridor 
and increase the complexity of sediment assessment, making traditional assessment approaches 
less suitable for this setting.

\medskip

\textbf{Our goal is to develop a more economical and efficient method to assess sediment contamination levels} 
specifically tailored to the Huron-Erie corridor, with the potential for broader application 
to nearby aquatic ecosystems. The foundational idea is inspired by the work of Jian 
(Zhang 2008)~\cite{Zhang2008}, who investigated the composition of zoobenthic 
communities to infer contamination in this region.

\textbf{Building on Jian's general framework}, we aim to both enhance the original methodology and 
incorporate recent advancements—enabled by improved computational resources and 
emerging context-specific analytical techniques. Through this effort, we seek to create a 
more adaptive, data-driven approach to infer aquatic condition from zoobenthic measurements in complex freshwater systems.
\clearpage
\section{Research Objectives}

The goal of this study is to build a zoobenthic community indicator(ZCI) of sediment contamination 
in a large aquatic ecosystem(e.g., SCDRS or Detroit River) by means of multivariate statistical analysis.
This work can be divided into five specific objectives with respects to aquatic ecology principles and
statistical methodologies:

\begin{enumerate}
    \item Build quantitative composite measures of sediment contamination levels and zoobenthic community composition.
        \item Control for natural variability in community composition and isolate anthropogenic effects on community composition.
        \item Incorporate spatial features to account for potential spatial heterogeneity in community composition.
        \item Model the relationship between the sediment contamination levels and zoobenthic community composition
        with piecewise quantile regression and extend it to a bioindicator (ZCI) of sediment contamination level.
        \item Evaluate the indicator power and robustness of the developed ZCI with respect
        to sample size and other relevant global factors that can influence the estimates.
\end{enumerate}

These listed objectives are made in a tentatively sequential order. Some objectives have been well-supported by
existing researches while others are more exploratory. Objectives \textcolor{blue}{1, 2} and \textcolor{blue}{4},
form a good benchmark of building the zoobenthic community indicator and are supported by a well-established
\textbf{Multivariate Community Indicator Framework}
\footnote{Its details are provided in following content and in the Methodology section}
\cite{Brazner2007, Kovalenko2014, Host2019, Ciborowski2005ZoobenthicIndicators,Zhang2008}
amenable to application of a piecewise quantile regression model.
Objectives \textcolor{blue}{3} and \textcolor{blue}{5} require exploratory investigations, 
and will serve to enhance and increase understanding of the indicator's power and sensitivities.

The following subsections disccussed the proposed general approach, highlighting the focuses where possible contributions
can be made.

\subsection{Multivariate community indicator framework}

Multivariate community indicator framework is a comprehensive approach for achieving objectives \textcolor{blue}{1, 2} and \textcolor{blue}{4}, 
we will build it with possible improvements.
Three data matrices (Zhang 2008 \cite{Zhang2008}; Farara and Burt 1993 \cite{Farara1993}; Wood 2004 \cite{Wood2004}) will be used for it:
the taxa matrix (biological response), the environmental covariate matrix, and the stressor (chemical) matrix.
Furthermore, quantitative indices that summarize sediment contamination and zoobenthic community composition
will be designed and applied meanwhile controlling the natural variability. 
Its basic steps are summarized as follows:

\begin{enumerate}
    \item \textbf{Measure sediment contamination and select reference sites}
    
    \emph{Motivation: establish a defensible composite measure that reflects a contamination gradient and identify 
    least disturbed sites that serve as reference locations.}

    Perform Principal Component Analysis (PCA) on elements of the stressor matrix;
    filter and interpret qualified pollutant Principal Components (PCs) from which to form a
    composite contamination score. Designate sites with the scores indicative of minimally-disturbed
    levels as reference sites from which to identify their putative "reference condition" assemblages.

    \item \textbf{Predict 'reference condition' community composition across environmental gradients} 
    
    \emph{Motivation: predict the variation in community composition as controlled by environmental covariates
    to disentangle the pollution-driven deviation in the community composition.}

    Perform hierarchical agglomerative cluster analysis of the biota of reference-site samples to identify
    distinct groups of taxa; train a discriminant model to predict the biological assemblages present at 
    a location on the basis of the location’s environmental conditions. 
    Thus, sites having the same biological assemblage are considered environmentally homogeneous.

    \item \textbf{Measure community composition (ZCI) for environmentally homogeneous sites}

    \emph{Motivation: construct a numerical index for each biological assemblage to summarize
    community composition for sites with similar environmental conditions.}

    Within each predicted cluster, apply ordination (e.g., PCA/NMDS) to summarize taxa information;
    derive the measurement results as ZCI with defined scoring rules.
    Sites deviating from the reference site centroid are considered disturbed by pollution,
    they compromise the disturbance-relevant ZCI results.

    \item \textbf{Quantify the relationship between contamination level and ZCI}
    
    \emph{Motivation: quantify how community composition responds to contamination levels}

    Fit regression models of ZCI deviation versus contamination score to capture deterministic relationships 
        and stochastic variation that shapes the conditional distributional response.
        Determining a suitable relationship pattern, e.g., smooth linear or breakpoint-based piecewise,
        is the key to the first step.
        If the latter exists, the following tasks will be to identify (1) the level of contamination at which the breakpoint(s) occur and
        (2) the benchmark(s) of ZCI representative of these crucial contamination levels.
\end{enumerate}



\subsection{Spatial heterogeneity test and incorporation}

Objective \textcolor{blue}{3} is to ensure the unmodelled spatial patterns do not confound the regression results.
Ecological data often exhibit spatial autocorrelation—nearby sites tend to have similar communities—so
failing to include spatial structure can lead to biased estimates and inflated error rates.
The principal coordinates of neighbour matrices (PCNM) method will be applied to transform spatial distances among sites into orthogonal spatial
predictors that can be incorporated into regression or canonical analyses \cite{Borcard2002PCNM, Dray2006SpatialEigenfunction, GriffithPeresNeto2006SpatialFiltering}.

PCNM eigenvectors from the Euclidean distance matrix will capture spatial structure at different scales \cite{Borcard2002PCNM, GriffithPeresNeto2006SpatialFiltering}.
For each environmental cluster of sample sites, we will test for spatial heterogeneity by regressing the ZCI 
derived for that group of sites against PCNM vectors 
to identify significant spatial predictors not explained by environmental variables.
Selected PCNM variables will be incorporated as covariates in quantile regression models to control spatial autocorrelation.
% Variance partitioning will help control the correlation among contamination levels across space,
% improving the accuracy in estimating the ZCI-contamination relationship.

\subsection{Piecewise quantile regression for breakpoint in quantile relationship}

Objective~\textcolor{blue}{4} is to use a piecewise quantile regression to model the relationship 
between sediment contamination levels and a ZCI,
which identifies breakpoints in the ZCI-contamination relationship within the multivariate community indicator framework. 
Piecewise quantile regression (PQRM) is chosen for its ability to estimate conditional quantiles 
of a response without assuming a specific parametric form \cite{Cade2003,Huang2017}, 
to model the entire conditional distribution rather than just the mean \cite{Cade2003},
and to estimate parameters that are robust to outliers or heteroscedastic variance \cite{Huang2017}.
In the ecological context, quantile regression can reveal abrupt changes in zoobenthic community 
composition associated with specific contamination levels at different quantiles 
(e.g., high quantile representing sensitive taxa)
and detect breakpoints that indicate ecological thresholds \cite{Jabed2020,Spake2022,Daily2012}.

It is informative to detect threshold(s) in contamination levels across which the slope of the ZCI–contamination relationship 
changes abruptly, indicating points where benthic communities change their way in responding to contamination. 
Higher quantiles may reveal steeper changes than median quantiles, highlighting vulnerable taxa. 
Comparing breakpoints across environmentally distinct clusters (if the comparison across environmental gradients is feasible)
will show whether thresholds vary among environmental gradients.


\subsection{Indicator power and robustness with respect to sample size}
The goal of Objective~\textcolor{blue}{5} is to evaluate how reliably
the ZCI detects sediment contamination gradients under varying sampling restrictions and conditions.
Two primary targets can be (1) exploring the minimum size of training data that supports the reliable estimation 
of key parameters (e.g., slopes, breakpoints)\cite{Spake2022} in the ZCI–contamination relationship; 
(2) how large a test set should be used to determine whether a new site falls above or below 
particular threshold(s) with acceptable probabilistic confidence when it truly does so.


Robustness analyses are essential because bioindicator performance can deteriorate when sample sizes are small, variance components are poorly estimated, or site selection produces hidden pseudoreplication \cite{Hurlbert1984Pseudo}. Power and precision directly influence management credibility; underpowered indicators risk Type II errors (failing to flag degraded conditions) while unstable estimates inflate Type I error rates in threshold detection \cite{Osenberg1994ImpactPower,Fairweather1991MonitoringPower}.

The analytical approach is tentatively to use bootstrap resampling and subsampling to evaluate ZCI robustness.
Bootstrap resampling may generate sampling distributions for key parameters (slopes, breakpoints, pseudo-$R^2$),
while subsampling could construct precision curves by repeatedly selecting subsets of sites across sample sizes.
Power analysis will potentially simulate datasets under null and alternative hypotheses to compute detection power
for contamination effects and threshold shifts. Spatial autocorrelation may need to adjust effective sample sizes,
and breakpoint reliability could be assessed through confidence interval width criteria or other suitable metrics.


Results will identify optimal sampling design (sites per cluster) and provide decision-support tables
linking confidence levels to required sampling effort. Higher quantiles may require larger samples due to greater dispersion,
while also elevating the potential concerns regarding spatial autocorrelation.






\section{Data Description}

\subsection{Data Collection}

The data used in this program is provided by Dr. Ciborowski, collected and 
processed by Zhang.
It consists of data from
three separate surveys conducted in: 1991, 1999 and 2004 (Zhang \cite{Zhang2008}; Farara and Burt 1993 \cite{Farara1993}; Wood 2004 \cite{Wood2004}),
all following the same field protocols
\footnote{
These sampling locations were determined prior to 
fieldwork by a stratified random sampling design to ensure representative coverage.}.

The 2004 data set was collected by Zhang \cite{Zhang2008} and principally across the entire SCDRS zone.
The information collected includes sample location information (longitude and latitude),
16 zoobenthic taxonomic variables, 5 environmental
variables (e.g., temperature, pH, dissolved oxygen), and 30 stressors representing trace metals,
polyaromatic hydrocarbons (PAHs).
The data from two previous studies, which collected data from the Detroit River zone in 1991 and 1999
(Farara and Burt 1993 \cite{Farara1993}; Wood 2004 \cite{Wood2004})—were compiled and incorporated into the 2004 data.
This combination enhances the dataset's robustness by providing a more comprehensive perspective on
the benthic community dynamics, environmental conditions, and sediment contamination across the entire Corridor over time.


Given the temporal and spatial distribution of sampling across three survey years,
\textbf{StationID serves as the primary key for data integration and site identification.}
{Each StationID uniquely identifies a sampling site in both temporal and spatial dimensions, 
meaning observations with the same StationID represent identical location-time combinations 
from the three survey years (1991, 1999, 2004).}

\subsection{Prepare a complete sub-dataset for preliminary analysis}

Taxonomic, environmental, and stressor data for sampled sites were originally stored in three separate tabular files.
For preliminary analysis, we made a quick check on the data quality and completeness
\footnote{By the submission 
of the proposal the latest check was done on July 24, 2025.}.

\begin{figure}[!h]
    \centering
    \includegraphics[width=1\textwidth]{../results/different_data_sampling_locations.png}
    \caption{\textit{Map of the SCDRS showing the distribution of locations for which contaminants(left - red points),
environmental covariates (centre - blue points) and zoobenthic data (right - green points) were collected.
Different samples were collected in both temporal and spatial dimensions}}
    \label{fig:different_data_sampling_locations}
\end{figure}

Figure \textcolor{blue}{\ref{fig:different_data_sampling_locations}} shows the numbers
and locations of observations in each dataset.
Note that some locations may be sampled in different
years and only a StationID
\footnote{StationID uniquely identifies a sampling site in both temporal and spatial dimensions.}
(rather than a geographical location) can be
used to confirm the identity of a sampling site.
The three datasets with different types of information (environmental, taxonomic, and stressor) differed in
sample sizes of their observations (row counts).
The stressor dataset contains the fewest observations (104), while the taxonomic and environmental datasets
contain more observations (289 or more).
To prepare a complete dataset for preliminary analysis,
the three datasets were merged by StationID using inner-join operations,
resulting in a comprehensive dataset with 104 observations containing taxonomic, 
environmental, and stressor data across the Lake Huron-Erie Corridor.

This sample size misalignment will be resolved when complete data becomes available in the near future.
Since sample size is the primary difference between current and forthcoming datasets,
the preliminary analysis framework is designed for easy scalability
when additional data becomes available.

\subsection{Large River Case Study: extra water velocity data in Detroit River}
Previous work \cite{Zhang2008} identified limitations in model performance due to insufficient environmental variable coverage.
In a Detroit River-focused analysis \cite{Zhang2008},
a new environmental variable—bottom \textbf{water velocity}—was added,
which was derived by Dr. S. Reitsma using a three-dimensional water flow model.
The focused analysis demonstrated that water velocity is a critical environmental variable for
controlling environmental variation.
Therefore, the Detroit River will serve as a focused study area with water velocity data included,
and it will be conducted once complete stressor data becomes available.

\subsection{Environmental attributes and samples}

Across the three SCDRS surveys, 
location information (longitude and latitude recorded via GPS readings) and 5 environmental attributes
were measured at each sampling site.

\textbf{Temperature ($^\circ$C)} and \textbf{dissolved oxygen concentration ($mg/L$)} were measured using a Hydrolab multimeter. 
\textbf{Water depth ($m$)} was recorded from the Ponar rope.
\textbf{Loss on ignition (\%)} and \textbf{median particle size (phi-units)} were determined during sediment 
processing but are treated as environmental attributes due to their fundamental roles in habitat
characterization (details on their analysis are provided in a later subsection).
% \footnote{Although loss on ignition and median particle size are derived from sediment samples, 
% they are considered environmental variables because they reflect essential physical and chemical habitat features.}
\textbf{Water velocity ($m/s$)} was estimated for the Detroit River area across the three surveys,
displayed in the Table \textcolor{blue}{\ref{tab:environmental_attributes}} along with other cross-corridor attributes.



\begin{table}[htbp]
\centering
\caption{Environmental Attributes}
\label{tab:environmental_attributes}
\begin{tabular}{|>{\centering\arraybackslash}m{4cm}|>{\centering\arraybackslash}m{11cm}|}
\hline
\textbf{Attribute} & \textbf{Ecological Relevance} \\[0.5em]
\hline
Temperature ($^\circ$C) & Controls metabolic rates and organism distribution patterns. \\[0.5em]
\hline
Dissolved Oxygen Concentration ($mg/L$) & Determines survival and excludes oxygen-sensitive taxa when low. \\[0.5em]
\hline
Water Depth ($m$) & Affects light penetration and benthic habitat availability. \\[0.5em]
\hline
Loss on Ignition (\%) & Indicates organic matter content and food availability for benthos. \\[0.5em]
\hline
Median Particle Size (phi-units) & Determines substrate stability and habitat suitability for taxa. \\[0.5em]
\hline
Water Velocity \(^*\) ($m/s$) & Controls flow regime and determines which taxa can colonize sites. \\[0.5em]
\hline
\end{tabular}
\begin{flushleft}
\footnotesize
\textit{(i)} \(^*\) Water velocity was estimated from the model by Dr. S. Reitsma (Detroit River area only, 225 observations) \cite{Zhang2008}.
\\ \textit{(ii)} Other environmental attributes were measured at all sampling sites across the entire survey area (311 observations).
\\ \textit{(iii)} All three surveys (1991, 1999, 2004) followed identical collection protocols and were merged for comprehensive temporal analysis.
\end{flushleft}
\end{table}

These attributes are commonly used to describe baseline environmental conditions in aquatic habitats,
as they are primarily governed by natural physical processes that 
influence taxonomic composition \cite{Davies2006, Turner1989PatternProcess}.

By including these variables as covariates to partially partition the zoobenthic community composition,
we can partially control for natural variation contributed by habitat characteristics, thereby 
isolating the effects of anthropogenic stressors in subsequent analyses 
of community composition patterns.

% However, it is important to note 
% that certain attributes (e.g., organic matter content or temperature) may also be
% indirectly influenced by anthropogenic activities in some settings. 
% Care was taken to interpret their roles in the context of site history and
% land use, but in general, these variables serve as key descriptors of the 
% underlying habitat conditions independent of direct contamination or disturbance.


\subsection{Taxonomic attributes and samples}
The zoobenthos were collected with a Ponar grab sampler. 
After considering the fullness of each grab and the removing of fine materials,
the team applied multiple grabs at each site until a total volume of 2\(L\) sediment was collected.
The sediment samples for organic and metals analysis were preserved in corresponding professional containers, 
all these samples were stored frozen.

One zoobenthic sample replicate from each site was randomly selected and processed, 
while the other two were archived. Samples were sieved into size fractions (4 mm, 1 mm, 0.5 mm, 0.25 mm), 
then elutriated to separate lighter detritus and animals from inorganic sediments. 
Each fraction was sorted under a microscope and organisms were identified to the lowest 
possible taxonomic rank using standard keys. Zoobenthos were preserved in 70\% ethanol in 
labeled vials and archived at the University of Windsor\cite{Zhang2008}.
Immediately after the initial sorting of samples, ten samples were randomly selected to assess the sorting efficiency. 
One sample had a sorting efficiency of 91\%,
while the remaining samples had efficiencies of 96\% or higher.

Specifically, there were 16 taxa recorded from the sediment samples, as shown in the table \textcolor{blue}{\ref{tab:taxonomic_variables}}.
According to their creature characteristics and preferred habitat, 
these taxa can be gently divided into three groups according to their preferred habitat.

\begin{table}[htbp]
\centering
\caption{Benthic Taxa and Preferred Habitat Features}
\label{tab:taxonomic_variables}
\begin{tabular}{|>{\centering\arraybackslash}m{3cm}|>{\centering\arraybackslash}m{5cm}|>{\centering\arraybackslash}m{4cm}|}
\hline
\textbf{Taxa} & \textbf{Explanation} & \textbf{Preferred Habitat} \\
\hline
Nematoda         & Roundworms                   & \multirow{6}{*}{Broad} \\
Chironomidae     & Non-biting midges (larvae)   &  \\
Ceratopogonidae  & Biting midges                &  \\
Amphipoda        & Small crustaceans (scuds)    &  \\
Acari            & Aquatic mites                &  \\
Hydrozoa         & Small predatory animals      &  \\
Gastropoda       & Snails and slugs             &  \\
\hline
Oligochaeta      & Aquatic segmented worms      & \multirow{6}{*}{Depositional zone} \\
Hexagenia        & Mayfly genus (larvae)        &  \\
Dreissena        & Zebra/quagga mussels         &  \\
Hirudinea        & Leeches                      &  \\
Turbellaria      & Flatworms                    &  \\
Sphaeriidae      & Fingernail clams             &  \\
\hline
Caenis           & Mayfly genus (larvae)        & \multirow{3}{*}{Erosional zone} \\
Hydropsychidae   & Net-spinning caddisflies     &  \\
Other Trichoptera& Other caddisfly families     &  \\
\hline
\end{tabular}
\end{table}

\begin{itemize}
    \item \textbf{Broad habitat:} Characterized by a wide range of environmental tolerance. Species in this group can inhabit both depositional and erosional areas, adapting to variable flow velocities, sediment types, and oxygen levels.
    
    \item \textbf{Depositional zone:} Areas with low to moderate water velocity where fine sediments (silt, clay, and organic matter) settle. These habitats often exhibit higher organic content and reduced oxygen penetration, favoring taxa adapted to softer substrates and potentially more enriched nutrient conditions.
    
    \item \textbf{Erosional zone:} Areas with higher flow velocity, coarser substrates (gravel, cobble, or sand), and well-oxygenated water. These habitats support taxa adapted to cling or attach to stable surfaces and withstand stronger currents.
\end{itemize}





\subsection{Stressors attributes and samples}

Sediment samples from each site were thoroughly mixed to ensure homogeneity. The homogenized samples were then split into separate portions for different analyses, including median particle size, total organic carbon (TOC), organic contaminants, and metals.

\begin{itemize}
    \item \textbf{Particle Size:} Median particle size analysis was performed by sieving dried sediment through a series of sieves of decreasing mesh size. Each size fraction was weighed and described using phi units ($\phi = -\log_2 d$), where $d$ is particle size in mm.
    \item \textbf{Total Organic Carbon (TOC):} Sediment TOC(\%OC) was determined using loss on ignition (LOI). Pre-weighed, dried sediment samples were combusted at 450$^\circ$C for 24 hours, and organic carbon was determined gravimetrically by subtracting the remaining mass.
    \item \textbf{Organic Contaminants:} The concentrations of organic contaminants (including 1245-TCB, 1234-TCB, QCB, HCB, OCS, p,p'-DDE, p,p'-DDD, mirex, Heptachlor Epoxide, total PCB)
     were measured using a gas chromatograph equipped with a 63Ni electron capture detector, following standard operating procedures.
    \item \textbf{Metals:} Metal concentrations (including Al, As, Ca, Cd, Co, Cr, Cu, Fe, Mn, Ni, Pb, and Zn) were analyzed using an Inductively Coupled Plasma Optical Emission Spectrophotometer (ICP-OES). For total mercury (Hg), an atomic absorption spectrophotometer (AAS) was used with a vapor generation accessory for increased sensitivity. Liquid samples were introduced into the instrument for metal analysis.
\end{itemize}

\begin{table}[htbp]
\centering
\caption{Sediment Stressors and Classifications}
\label{tab:stressors}

\renewcommand{\arraystretch}{1.2}
\begin{tabular}{|>{\centering\arraybackslash}m{2.5cm}|>{\centering\arraybackslash}m{7cm}|>{\centering\arraybackslash}m{3cm}|}
\hline
\textbf{Variable} & \textbf{Description} & 	\textbf{Type} \\
\hline
\multicolumn{3}{|c|}{\textbf{Metals (mg/kg sediment)}} \\
\hline
Al & Aluminum concentration (nontoxic) & Earth Element \\
Ca & Calcium concentration (nontoxic) & Earth Element \\
Fe & Iron concentration (nontoxic) & Earth Element \\
K & Potassium concentration (nontoxic) & Earth Element \\
Mg & Magnesium concentration (nontoxic) & Earth Element \\
Na & Sodium concentration (nontoxic) & Earth Element \\
As & Arsenic concentration (pollutant) & Trace Metal \\
Bi & Bismuth concentration (pollutant) & Trace Metal \\
Cd & Cadmium concentration (pollutant) & Trace Metal \\
Co & Cobalt concentration (pollutant) & Trace Metal \\
Cr & Chromium concentration (pollutant) & Trace Metal \\
Cu & Copper concentration (pollutant) & Trace Metal \\
Hg & Mercury concentration (highly pollutant) & Trace Metal \\
Mn & Manganese concentration (pollutant) & Trace Metal \\
Ni & Nickel concentration (pollutant) & Trace Metal \\
Pb & Lead concentration (pollutant) & Trace Metal \\
Sb & Antimony concentration (pollutant) & Trace Metal \\
V & Vanadium concentration (pollutant) & Trace Metal \\
Zn & Zinc concentration (pollutant) & Trace Metal \\
\hline
\multicolumn{3}{|c|}{\textbf{Organic Carbon (mg/kg sediment)}} \\
\hline
\%OC & Organic carbon content & Binding agent \\
\hline
\multicolumn{3}{|c|}{\textbf{Organic Contaminants (mg/kg sediment)}} \\
\hline
1245-TCB & 1,2,4,5-Tetrachlorobenzene (hydrocarbon pollutant) & Industrial compound \\
1234-TCB & 1,2,3,4-Tetrachlorobenzene (hydrocarbon pollutant) & Industrial compound \\
QCB & Pentachlorobenzene (hydrocarbon pollutant) & Industrial compound \\
HCB & Hexachlorobenzene (hydrocarbon pollutant) & Industrial compound \\
OCS & Octachlorostyrene (hydrocarbon pollutant) & Industrial compound \\
p,p'-DDE & Dichlorodiphenyldichloroethylene (pesticide) & Organochlorine \\
p,p'-DDD & Dichlorodiphenyldichloroethane (pesticide) & Organochlorine \\
mirex & Mirex (pesticide) & Organochlorine \\
Heptachlor Epoxide & Heptachlor Epoxide (pesticide) & Organochlorine \\
total PCB & Total polychlorinated biphenyls & Sum of all PCBs \\
\hline
\end{tabular}
\end{table}

Quality assurance and chemical analyses were performed in collaboration with the Great Lakes
Institute for Environmental Research (GLIER) at the University of Windsor\cite{Zhang2008}.
Among the chemical variables analyzed, major earth elements (Al, Ca, Fe) are generally non-toxic at typical environmental concentrations, reflecting natural sediment composition. 
However, elevated levels from industrial activities can make them potential stressors, 
leading to difficulties in assessing their impacts due to naturally high background concentrations.

In contrast, trace metals (As, Bi, Cd, Co, Cr, Cu, Hg, Mn, Ni, Pb, Sb, V, Zn) are 
anthropogenic pollutants that bioaccumulate in sediments and cause toxicity in benthic organisms.

Persistent organic pollutants (PCBs, QCB, HCB, OCS, p,p'-DDE, p,p'-DDD, mirex, 
Heptachlor Epoxide) from industrial activities and pesticide use persist in sediments 
and accumulate in food webs, causing chronic effects on aquatic organisms.

Percent organic carbon (\%OC) from decomposed organic matter influences contaminant 
binding and bioavailability, modulating the ecological impact of other pollutants.

By distinguishing between natural major elements (Al, Ca, Fe), trace elements (Co, Mn, Sb, V), 
toxic metals (As, Cd, Cr, Cu, Ni, Pb, Zn), the highly toxic mercury (Hg), organic matter content (\%OC), 
and persistent organic pollutants (including pesticides and industrial compounds), 
these measurements enable a more comprehensive assessment of the sediment stress level and 
its ecological implications for the benthic community.



\section{Literature Review(\textcolor{blue}{to majorly rewrite})}

% \subsection{Aquatic Integrity Assessment and Taxonomic Composition Response}

\subsection{\textcolor{blue}{Sediment Contamination Evaluation Methods}}
\textcolor{blue}{Majorly to talk about the common sediment contamination evaluation methods,
and distinguish which methods are suitable across various environmental conditions.
The generality of the assessment method can support a good stress level-environmental distribution in the data,
which is crucial to make sure enough data points with fixed stress level across the environmental values.}

\subsection{\textcolor{blue}{Taxa composition clustering with minimal (fixed) stress level}}
\textcolor{blue}{Majorly to talk how this clustering can help to identify the taxa composition patterns,
which can be used as predicted values of the discriminant function analysis.}

\subsection{\textcolor{blue}{Discriminant Function Analysis of environmental variables for taxa composition clustering}}

\subsubsection{Aquatic Integrity by Sediment Contamination Evaluation}

% \textcolor{blue}{What is aquatic integrity?}  

Aquatic integrity refers to how well an ecosystem maintains its structure and function under both natural and human pressures.
 It encompasses the ability of aquatic systems to support and sustain a balanced, adaptive community of organisms having a species
  composition, diversity, and functional organization comparable to natural habitats within a region.

% \textcolor{blue}{What is sediment?}  

In aquatic ecosystems, sediments act as long-term pollutant archives. They accumulate and retain contaminants over time—especially 
heavy metals, pesticides, hydrocarbons, and other industrial or urban pollutants. Therefore, sediments exert a considerable influence 
on the biological health of benthic organisms. However, it is often difficult to determine whether metal accumulation is due to
 natural processes or anthropogenic sources~\cite{Birch2007}.

% \textcolor{blue}{What makes sediment useful for aquatic integrity assessment?}  

Within such pollutant archives, anthropogenic contaminants accumulate through various pathways, including industrial 
discharges and agricultural runoff. Many of these pollutants—such as cadmium, lead, and mercury—are toxic to benthic organisms
 even at low concentrations. High contaminant levels are frequently associated with reduced biodiversity and shifts in community composition. 
 These biological responses provide the foundation for using zoobenthic measurements to inversely infer sediment contamination levels and, 
 in turn, evaluate the ecological integrity of aquatic systems.

However, anthropogenic pollution is not the only driver of changes in biological communities. Natural environmental 
variability—including sediment texture, organic matter content, and hydrological conditions—also plays a significant 
role in shaping benthic communities. As such, sediment contamination alone cannot fully explain or represent ecological 
integrity. To account for the full picture, environmental conditions must also be considered.

Nonetheless, in this program, we emphasize the controllable component of aquatic degradation—human-induced 
pollution—as our primary focus. While environmental variability is acknowledged, it is not the central concern of this study.
Our goal is to develop a model that assesses ecological condition by leveraging sediment contaminant data and linking it to 
zoobenthic community structure, with the understanding that it may reflect only part of the full ecological integrity.

% \textcolor{blue}{Mention some specific examples of sediment contaminant indices.}  

Frequently used evaluation methods include contaminant index-based approaches such as the Enrichment Factor (EF) and the
Geoaccumulation Index (Igeo). These indices provide pollution scores relative to known background or reference values, 
often focusing on individual or selected chemical elements~\cite{Birch2022Review}. However, such methods typically ignore 
interactions between pollutants and rely on accurate background concentrations—which may be difficult to obtain or define.
Furthermore, these approaches are less effective when analyzing large-scale datasets with many chemical elements and spatially distributed sampling sites.

% \textcolor{blue}{mention the PCA method}

Another type of evaluation method is Principal Component Analysis(PCA) based methods.
Such methods are multivariate and data driven, considering all chemical variables and revealing their 
interaction pattern by dimensionality reduction. 
Additionally, they do not need for background/reference values, which saves the resources for identifying 
these benchmarks. By trade off, PCA methods may provide less intuitive scores(assessments) that lack
direct interpretation, and the absence of benchmarks constrains the clear or comparable assessment results,
no "moderately" or "severely" polluted sites can be identified as in the index methods.

In conclusion, both groups of methods serve different but complementary purposes:
indices for direct contamination assessment, and PCA for exploring patterns or creating composite 
indicator.

% \textcolor{blue}{mention the natural variabilities in shaping the integrity}

However, the integrity of an aquatic site is determined not only by the anthropogenic impacts, but also by the natural pressures,
ignoring natural pressures can make the assessment results biased, like over- or underestimating ecological degradation.
The potential natural variabilities that shape the aquatic integrity include:
geology, flow regime, temperature, elevation and so on.
Consequently, a naturally metal-rich geology site might appear "degraded" but actually be natural,
and a site with no pollution but naturally poor biological diversity
\footnote{Biological diversity is not an environmental attribute, but a biological response variable.}
might be wrongly flagged as "degraded" in
the inference process.
Therefore, quantifying anthropogenic pollution through sediment contamination data mainly 
aims to assess the anthropogenic influence, and controlling for natural variation
is necessary to assess the aquatic integrity through such quantification on pollution.


\subsubsection{Aquatic Integrity by Biological Condition Gradient}
% \textcolor{blue}{mention the represent of another assessment method: BCG}

Another type of assessment method is Biological Condition Gradient (BCG),
which is different from the contaminant evaluation in its conceptual basis, data sources and focus.

BCG starts with a stress-response framework, categorizing sites into 6 biological condition levels,
uses biological metrics(which the contaminant evaluation do not) to measure the biological responses 
and other ecological responses to the stressors and finally makes an integrative measure of 
ecosystem integrity.
One of the key advantages of BCG is its widely applicable feature over a relatively large range
\footnote{Large ranges: regional level(over several states or provinces), country level, continent level.}
of aquatic ecosystems, which makes the biological condition to be interpreted independently of 
assessment methods\cite{Davies2006}. However, it needs to build reference conditions from minimally disturbed sites,
and deviation from these benchmarks(reference conditions) reflects biological degradation,
which requires empirical data and expert judgment to assess the biological integrity.

Considering my research objectives and the available data sources, 
the contaminant evaluation are more suitable for my work, which reflects potential stress but not 
whether ecosystems are responding biologically. Such conceptual properties make the inference 
for degree of pollution through \textbf{biological responses} possible and avoid laking the 
biological information to the input data of the inference model.

\subsection{Ecological Thresholds Detection and Inference}

\subsubsection{Ecological Thresholds Existence and Application Scope}

% \textcolor{blue}{what is ecological threshold}

Ecological thresholds are points at which a relatively small change in an 
external condition(like pollution, or nutrient level) causes a rapid and 
significant change in ecosystem structure or function.

The concept of ecological thresholds emerged in the 1970's from the idea that 
ecosystems often exhibit multiple 'stable' states, depending on environmental conditions\cite{Holling1973}(Holling 1973, as cited in Groffman et al. 2006).
These stable states that are separated by thresholds can be long-lasting conditions 
that an ecosystem can exist in - such as clear-water lake with abundant vegetation versus 
a turbid, algae-dominated lake, with different ecological structures, functions and species compostion.

The shifts between these states occur when thresholds are crossed. 
Some shifted states may arise naturally and are not harmful to human societies
or the surrounding environment, but others may be, leading to the loss of both 
economic and ecological value.
For those potential shifts toward undesirable states that may or will cause 
such losses, detecting and inferring thresholds is economically and ecologically important, 
as it helps guide management policies and prevent potential degradation. 
Additionally, the identified thresholds can help set restoration targets,
beyond which preservation efforts are more likely to support long-term ecological 
goals—especially when the recovered state is near a threshold and at risk of degrading again.
Therefore, interest in ecological threshold has grown in both ecological management and 
restoration fields, with the aim of maintaining or restoring ecosystems in a desired state.

% \textcolor{blue}{talk the commonly existing thresholds in aquatic ecosystems}

Turning to the scope of ecological threshold analysis, there are three main ways that threshold concepts have been applied in ecology:
(1) analysis of dramatic and surprising "shifts" in ecosystem state; 
(2) the determination of "critical loads"
(3) analysis of "extrinsic factor thresholds"
\cite{peterson2006ecological}. Their explanations are summarized in Table \ref{tab:thresholds_application_scopes}.

The three scopes have respective preferred applications but they are not mutually exclusive. In some studies,
they are used together to provide a more comprehensive understanding of ecological thresholds, extending the
threshold application to multiple aspects of research and management.

\begin{table}[ht]
\centering
\caption{Three main ways threshold concepts are applied in ecology~(Peterson, etc. 2006 \cite{peterson2006ecological}).}
\label{tab:thresholds_application_scopes}
\renewcommand{\arraystretch}{1.8} % Increase row height for vertical centering
\begin{tabular}{>{\centering\arraybackslash}p{4.5cm} >{\centering\arraybackslash}p{9.5cm}}
\hline
\textbf{Application Aspect} & \textbf{Meaning} \\
\hline
Shifts in ecosystem state & Analyzing dramatic and surprising changes in ecosystem condition caused by small changes in a driver. \\
\hline
Critical loads & Determining the pollutant level an ecosystem can absorb without experiencing a shift in state or function. \\
\hline
Extrinsic factor thresholds & Analyzing how large-scale variable changes affect relationships between drivers and responses at smaller scales. \\
\hline
\end{tabular}
\end{table}

My project aligns with two ecological threshold applications. 
It fits the \textit{shifts in ecosystem state} scope by using zoobenthic 
data to detect biological changes that signal ecosystem transitions. 
It also supports the \textit{critical loads} approach by 
linking stressor levels to community responses, estimating pollutant thresholds 
beyond which ecological integrity declines, while controlling for environmental
variation.

\subsubsection{Ecological Thresholds Detection and Inference Methods}
% \textcolor{blue}{Driven factors that cause ecological thresholds}

Both natural variables (e.g., temperature, altitude) and anthropogenic stressors (e.g., pollutant concentration)
can show threshold behavior, where crossing a  breakpoint in both these variables can lead 
to rapid shifts in ecosystem structure or function. 

% \textcolor{blue}{talk how bad the traditional regression methods are in detecting(not able, able but not bad) thresholds}

On top of that, the existence of thresholds often suggests 
the fact that the ecosystem is not linear in its response to external changes,
not only in the rate and direction of change, but also in the heteroscedatic 
nature and other properties. It challenges the classical regression methods 
that constrain the strict linearity assumption and urges the implmentation of
more flexible models that fit specific ecological contexts.

Typically, changes in conditional means are the exclusively focus
of many traditional regression methods, which may fail to distinguish
significant changes beyond the mean scope in heterogeneous distributions.
Some extended traditional methods, such as weighted least squares (WLS) is designed by assigning weights inversely 
proportional to the variance at each level, to address \emph{heteroscedasticity} where the variance is not constant.
However, heteroscedasticity is only one type of complexity in ecological data\cite{Cade2003}. Ecological relationships often involve \emph{multiple sources of variation}, such as:
\begin{itemize}
    \item \textbf{Unmeasured limiting factors:} Not all influential variables can be measured or included in the model, leading to variation in the response not explained by the predictors.
    \item \textbf{Heterogeneous responses:} Changes may occur primarily in certain portions of the response distribution (e.g., only in the upper quantiles), while mean trends remain flat.
    \item \textbf{Nonlinear and threshold effects:} Ecological thresholds may be apparent in certain quantiles or as abrupt changes not captured by mean regression models.
\end{itemize}
As a result, while methods like WLS can address heteroscedasticity, they may still fail to capture the 
full range of ecological relationships—especially when important ecological processes affect 
only a subset of the data or produce complex response patterns beyond simple mean-variance trends.

Quantile regression (QR) can be a desirable and practical solution to these challenges, benefiting from its 
less strict assumptions on parametrics and its ability to model the quantile-specific (\(\tau \in [0, 1]\)) relationship between 
variables\cite{Huang2017}. It provides alternatives to reveal limiting factors by fitting different quantiles of the response distribution, 
and does not require the heteroscedasticity assumption in relationships, making the exploration beyond the mean scope possible.
Additionally, the detection of nonlinear and threshold effects can be achieved by fitting piecewise quantile regression models,
capturing abrupt changes in the response distribution at specific quantiles.

% \textcolor{blue}{talk how these days people use new methods to detect thresholds, better than the traditional ones}

Horning (2013)\cite{Horning2012} explores the concept of ecological thresholds in the context of behavioral physiology,
and introduces the use of \textit{constraint lines}—the boundaries delimiting point
clouds in bivariate scatterplots—to reveal limiting factors and physiological constraints.
In his study, quantile regression were employed to analyze these constraint lines, this method losses the assumptions on parametrics
and therefore covers wider range of possible relationships between variables.

Cade and Noon (2003)\cite{Cade2003} makes a gentle discussion on quantile regression and its application in ecology,
highlighting the statistical properties and possible regulations in changes of estimated coefficients across 
different quantiles that a QR model may have under different complicated scenarios.
They discussed several representative examples of hidden effects in ecological data, which can not be captured by 
mean-regression models. Additionally, a quantile regression model was fitted on a linearly increasing heteroscedasticity case, 
and bootstrapping method was employed to estimate the confidence intervals of the coefficients, showing 
a relatively stable and reliable performance of a QR model in such cases.
They concluded that heteroscedasticity is the cause that leads to changes in the coefficients across quantiles,
and that the QR model can be a good choice to address this issue.
However, a "linearity" relationship between quantiles of the response and predictors should be the underlying factor 
that supports the use of QR models, because even a case of homoscedasticity (constancy in variance, not in distribution shape) 
can cause changes beyond the mean scope. 
Such a homoscedastic case—though rare in the real world—that results in changes beyond the mean further reinforces 
the need for quantile regression models in ecological data analysis.

% \textcolor{blue}{talk the potential factors that may influence the detection and inference we may meet during the work,}

Tests on accuracy and reliability of detected thresholds are necessary, several factors can affect these tests.
Based on the work of Daily et al. (2012)\cite{Daily2012}, these factors influence the detection quality with different degrees
in different contexts, including: 
\begin{itemize}
    \item Sample Size: Smaller sample sizes generally increase the rate of false threshold detection, where as 
    larger sample sizes improve the reliability of threshold estimates.
    \item Sample-Environment Distribution(SED): The frequency and distribution of samples across the 
    environmental gradient (SED) can greatly influence both the detection and the estimated location 
    of thresholds. Non-uniform or uneven SEDs can lead to biased or misleading results.
    \item Rate of Change: The actual rate of linear or nonlinear change in ecological response can interact 
    with statistical method properties, affecting detection outcomes.
    \item User-selected Model Parameters: The choice of model parameters-such as the quantile level(\(\tau\)) in QR, or 
    bandwidth in smoothing methods (e.g., sizer)-significantly impacts the detection rate and accuracy of threshold locations.
\end{itemize}

Correlatedly with the risks in false detection and/or inference, Spake et al. (2022)\cite{Spake2022} synthesizes evidence 
on threshold detection and emphasizes that the concept of scale is fundamental to understanding, quantifying, and interpreting ecological thresholds.
They organize the scale-dependence of threshold detection into four aspects, as shown in Table \ref{tab:scale_framework}.

\begin{table}[h!]
\centering
\caption{The Scale Framework in Ecological Threshold Detection (adapted from Spake et al., 2022)
(\textbf{Grain} refers to the smallest unit of measurement, affecting the resolution and 
detail of the data. \textbf{Extent} refers to the overall scope (area or time) of the
study, affecting the range of environmental or temporal gradients observed.)
}
\label{tab:scale_framework}
\renewcommand{\arraystretch}{2.2}
\begin{tabular}{|m{3.5cm}|m{11cm}|}
\hline
\centering\textbf{Scale Concept} & \centering\textbf{Description} \tabularnewline
\hline
\centering\textbf{Grain (Resolution)} & The size of the smallest unit of observation or measurement (e.g., plot size, pixel size, sampling interval); determines the level of detail captured. \tabularnewline
\hline
\centering\textbf{Extent} & The total area or duration covered by the study; determines the environmental or temporal gradient sampled and potential to observe thresholds. \tabularnewline
\hline
\centering\textbf{Organizational Level} & The biological or ecological hierarchy at which data are collected or analyzed (e.g., individual, population, community, ecosystem). \tabularnewline
\hline
\centering\textbf{Analytical Method} & The type of statistical or modeling approach used to detect thresholds, which can influence the sensitivity and interpretation of results. \tabularnewline
\hline
\end{tabular}
\end{table}


In my project, piecewise quantile regression model (PQRM) will be the major analytical method to detect the
potential thresholds between the stressors and the taxonomic composition, 
allowing for a nuanced understanding of their relationships. Such scale framework can inform my project together with the factors mentioned above by guiding decisions on sampling design,
data aggregation (if needed), and model selection, increasing the robustness and ecological relevance of threshold detection.




\subsection{Synthetic Data in Machine Learning for Ecological Assessment}

% \textcolor{blue}{In this section, i will review related references to talk 
% \begin{itemize}
%     \item What is synthetic data, and some statistical background to back up the idea of using it.
%     \item Why synthetic data is useful in ecological assessment, specifically in our case.
%     \item The common methods to generate synthetic data, including the statistical models and machine learning methods
% \end{itemize}
% }

References already in hands include:
\begin{itemize}
    \item \textit{'Small Data' for big insights in ecology (citation to be added)}
\end{itemize}

There are more references to be added this month, after reviewing and confirming their relevance to the thesis objectives.


\section{Methodology}

This methodology section details the overall analytical framework and specific techniques to be employed in this research.

The proposed methodology consists of the following key steps:

\begin{enumerate}
\item Dimension reduction on stressors that produce pollution scores and arranges samples in a low-dimensional pollutant space to reveal main synthetic stressor gradients.

\item Find and prepare least polluted sites, construct an estimator that estimates the counterfactual surface (minimal pollution level) of taxa community composition in environmental space.

\item Ordinate the taxa variables, construct comprehensive measurement for the taxa community.

\item Spatial heterogeneity test and geo-information incorporated for latent environmental variables.

\item Piecewise quantile regression models on ZCI and pollution scores, revealing relations at different quantile levels and detecting thresholds.

\item Step out of the framework, change global factors, like sample size and partitioning samples by biased environmental distribution. Study how the series of estimates would change accordingly.
\end{enumerate}

Various symbols are used throughout this section, including function names, vectors, and matrices. The following table summarizes these symbols:

\begin{table}[!h]
\centering
\caption{Summary of major mathematical symbols and their meanings, organized by subsection.}
\begin{tabular}{lll}
\toprule
\textbf{Subsection} & \textbf{Symbol} & \textbf{Meaning} \\
\midrule
\multirow{7}{*}{\parbox{3cm}{\centering Data Description and Sediment Contamination Assessment}} 
& $m$ & Number of sampled sites \\
& $X \in \mathbb{R}^{m \times 30}$ & Elemental concentration matrix (30 chemical elements) \\
& $E \in \mathbb{R}^{m \times 5}$ & Environmental variable matrix (5 variables) \\
& $T \in \mathbb{R}^{m \times 16}$ & Taxa abundance matrix (16 taxa) \\
& $s \in \mathbb{R}^m$ & Composite stressor score (from PCA) \\
& $p\%$ & Percentage of least stressed sites chosen as reference \\
& $I_{\mathrm{ref}} \in \mathbb{R}^m$ & Indicator: 1 = reference site, 0 = disturbed site \\
\midrule

\multirow{5}{*}{\parbox{3cm}{\centering Reference site clustering and Discriminant Function}} 
& $\mathcal{C}_K$ & Cluster label (taxa composition group) from reference sites \\
& $\hat{\mathcal{C}}_K$ & Predicted cluster label for disturbed sites \\
& $\mathcal{F}_{\mathrm{dis}}$ & Discriminant function mapping $E_{\mathrm{ref}}$ to $\mathcal{C}_K$ \\
& $\delta T_{i,j}$ & Taxa community structure relative to the scale of reference site \\
& $\delta X_{k,j}$ & Stress level relative to group-$k$ reference median \\
\midrule

\multirow{12}{*}{\parbox{3cm}{\centering
    Multivariate Gaussian modeling for constructing Zoobenthic Community Index (ZCI)
 }} 
& $\phi_{\mathrm{Hel}}$ & Hellinger transformation \\
& $\mathcal{R}_k$, $\mathcal{D}_k$ & Sets of reference and disturbed sites in group $k$ \\
& $\boldsymbol{\mu}_k$ & Mean taxa composition vector for group $k$ reference sites \\
& $\boldsymbol{\Sigma}_k$ & Covariance matrix of taxa composition in group $k$ \\
& $\lambda$ & Ridge regularization term \\
& $I_{16}$ & $16\times16$ identity matrix \\
& $\tilde{T}_{k,j}$ & Whitened deviation vector for site $j$ in group $k$ \\
& $\mathrm{ZCI}_{k,j}$ & Scalar Mahalanobis distance from reference centroid \\
& $\mathrm{ZCI}^{(\mathrm{diag})}_{k,j}$ & Diagonal approximation ignoring correlations \\
& $\mathrm{ZCI}^{(1)}_{k,j}, \mathrm{ZCI}^{(2)}_{k,j}$ & First two components of multi-dimensional ZCI \\
& $\mathrm{ZCI}^\star_{k,j}$ & 0--100 scaled ZCI score \\
& $V_k$ & PCA loading matrix from whitened reference deviations \\
\midrule

\multirow{4}{*}{\parbox{3cm}{\centering
    Spatial basis expansion for spatial predictors
}} 
& $x_i, y_i$ & Spatial coordinates of site $i$ \\
& $D \in \mathbb{R}^{m \times m}$ & Euclidean distance matrix(symmetrical) between sites \\
& $A \in \mathbb{R}^{m \times m}$ & Double-centered distance matrix from truncated distance matrix \\
& $S_{\text{sel}} \in \mathbb{R}^{m \times d}$ & Selected eigenvectors for explaining spatial variation (\(d < m\)) \\
\midrule

\multirow{7}{*}{\parbox{3cm}{\centering Quantile regression modeling}} 
& $\mathcal{F}_{k}$ & Regression function linking ZCI and spatial features to stress level \\
& $\delta X\mid Z, S_{\text{sel}}$ & Relative stress level given ZCI and spatial features \\
& $Q_{\delta X\mid Z, S_{\text{sel}}}^{(k)}(\tau \mid z, s)$ & Conditional $\tau$-quantile of $\delta X$ given ZCI and spatial features \\
& $f_{k, \tau} (z, s)$ & Quantile regression function for group $k$ at quantile $\tau$ \\
& $\kappa_m$ & Fixed breakpoint in piecewise regression \\
& $\gamma_{m,\tau}^{(k)}$ & Slope change after breakpoint $\kappa_m$ \\
& $\hat \theta_{\tau}^{(k)}$ & Estimated parameter for quantile regression \\
\midrule

\multirow{3}{*}{\parbox{3cm}{\centering Hypothesis testing for degradation}} 
& $F_{\delta X\mid Z}^{(k)}(x \mid z)$ & Conditional CDF of stress level given ZCI \\
& $x_k^*$ & Group-$k$ stress threshold for degradation classification \\
& $p$ & One-sided $p$-value for degradation test \\
\bottomrule
\end{tabular}
\end{table}


At the initial stage, the whole information about the sites can be shown in the matrix form:
\[
\left[
\begin{array}{ccc}
X & E & T
\end{array}
\right]
\in
\mathbb{R}^{m \times (30 + 5 + 16)}
\]
where
\(X \in \mathbb{R}^{m \times 30}\) (elemental concentrations),
\(E \in \mathbb{R}^{m \times 5}\) (environmental variables),
and
\(T \in \mathbb{R}^{m \times 16}\) (taxa abundances).

% orginal workflow framework figure
% \begin{figure}[!h]
%     \centering
%     \begin{minipage}{0.8\textwidth}
%         \centering
%         \includegraphics[width=\textwidth]{../results/workflow_of_general_workframe_part1.png}
%         \vspace{0.5em}
%         \includegraphics[width=\textwidth]{../results/workflow_of_general_workframe_part2.png}
%     \end{minipage}
%     \caption{Overview of workflow for the proposed methodology.}
%     \label{fig:workflow_of_general_workframe_parts}
% \end{figure}
\begin{figure}
\centering
\includegraphics[width=0.8\textwidth]{../results/ideas_visualization/overall_framework.png}  
\caption{\textit{Overview of workflow for the proposed methodology.}}
\end{figure}
\subsection{Find Reference Sites - Sediment Contamination Assessment}

\begin{tcolorbox}[colback=white!95!gray, colframe=white!20!orange, 
    title = \textbf{External Reference: Data-driven PCA-based Pollution Assessment}]
    Detailed methodology for sediment contamination assessment has been developed separately.
    The method report that is under development is available at:
    \begin{itemize}
        \item \href{https://drive.google.com/file/d/1L43eq924ydxNeLgRYH4EsfJd8nvNVOf9/view}
        {Click to access: Method report draft}
    \end{itemize}
\end{tcolorbox}


To assess the sediment contamination and find the reference sites, we need to compute a composite stressor score \(s\) based on the chemical data.

Let \(m\) be the number of sampled sites and \(X \in \mathbb{R}^{m \times 30}\) denote the matrix of chemical element concentrations (each row represents a site and each column represents an element).
Doing a principal component analysis (PCA) on \(X\) transforms it into a set of uncorrelated and high variation-loading components \(Z\).
On top of the \(Z\), we can select \(k\)(\(< 30\)) proper components with defined criteria to cover the most variation in pollutant elements
and define a composite stressor score \(s \in \mathbb{R}^{m}\)
by summing or weighting the selected raw principal components or their normalised variants:


\begin{enumerate}
    \item \textbf{Principal component reduction} – Apply principal component analysis (PCA) to \(X\).  PCA transforms \(X\) into a set of uncorrelated components \(Z = X W\), where \(W \in \mathbb{R}^{30\times k}\) holds loadings of the first \(k\) principal components.

    \item \textbf{Composite stress score} – Let \(Z = [\,z_1,\dots,z_k\,]\) with \(z_i \in \mathbb{R}^{m}\) the vector of scores on the \(i\)-th principal component. 
     Define a composite stressor score \(s \in \mathbb{R}^{m}\) by summing or weighting the selected raw principal components:
    \[
    s_j \;=\;\sum_{i=1}^k \omega_i\,z_{i,j}, \quad j \in \{1,\dots,m\}
    \]
    where \(z_{i,j}\) is the \(i\)-th PC score at site \(j\) and \(\omega_i\) are predetermined weights (often set to 1 when components contribute equally).
\end{enumerate}

After computing the composite stressor score, we can add this new information to the originally compound matrix:
\[
\left[
\begin{array}{cccc}
X & E & T & s
\end{array}
\right] 
\in
\mathbb{R}^{m \times (51 + 1)}
\]

This \(s\) vector is used to rank the sites with respect to 
the stress level and filter the pristine reference sites where
human impact is minimal or absent. 
Specifically, we rank sites by \(s\) and retain the least‑stressed \(p\%\) of the sites,
create an indicator vector \(I_{\text{ref}} \in \mathbb{R}^{m}\) where \(I_{\text{ref},j} = 1\) if site \(j\) is a reference site and \(I_{\text{ref},j} = 0\) otherwise.
\[
\left[
\begin{array}{ccccc}
X & E & T & s & I_{\text{ref}}
\end{array}
\right] 
\in
\mathbb{R}^{m \times (52 + 1)}
\]
To this sites with \(I_{\text{ref},j} = 1\), we assume they represent
the ideal taxa composition that is shaped by the given environmental conditions,
supported by the minimal or absent human disturbance.
\[
\left[
\begin{array}{ccccc}
X & E & T & s & I_{\text{ref}} 
\end{array}
\right]_{I_{\text{ref}} = 1}
\in
\mathbb{R}^{(p\% \times m) \times (53)}
\]
Therefore, in this submatrix, the \(X\) matrix only contains the minimal \(p\%\) stress levels across all sites,
controlling the human disturbance on the taxa composition.

\begin{figure}[!h]
\centering
\includegraphics[width=0.8\textwidth]{../results/ideas_visualization/operate_data_throughout2.png}
\caption{\textit{Visualization of how the new information is generated and integrated into the existing matrix.}}
\label{fig:p4_rules_for_data_operation}
\end{figure}

\subsection{Prepare metrics of “ideal” taxa composition - Cluster Analysis on References}
In the matrix
$\left[
\begin{array}{ccccc}
X & E & T & s & I_{\text{ref}} 
\end{array}
\right]_{I_{\text{ref}} = 1}
$, the set of taxa composition \(T_{\text{ref}}\) 
is assumed to be shaped by the environmental variables \(E_{\text{ref}}\),
a well-fitted regression model between the \(E_{\text{ref}}\) and \(T_{\text{ref}}\)
matrices can numerically tell us how the taxa composition is multidimensionally shaped by the environmental variables.

However, considering that the \(E_{\text{ref}} \in \mathbb{R}^{(p\% \times m) \times 5}\) only provides 5 environmental variables,
and there are many other potentially unmeasured and unmeasurable environmental factors, it is nearly impossible to train a fully quantitative
inference model that describes the below relationship well:
\[
\mathcal{F} : E_{\text{ref}}^{(p\% \times m) \times 5} \to T_{\text{ref}}^{(p\% \times m) \times 16}, \quad \text{poorly fitted model}
\]

To solve this issue, we can construct constrained predicted values \(T_{\text{ref}}^{q} (q < 16)\) from the \(T_{\text{ref}}\) matrix, which 
provides limited yet information about the community structure, so that the model $\mathcal{F} : E_{\text{ref}}^{(p\% \times m) \times 5} \to T_{\text{ref}}^{(p\% \times m) \times q}$
can be trained to avoid overfitting and improve its prediction performance.
\[
\mathcal{F} : E_{\text{ref}}^{(p\% \times m) \times 5} \to T_{\text{ref}}^{(p\% \times m) \times q}, \quad \text{improved fitted model}
\]
One ideal way to do this information compression is 
to partition the reference sites into \(K\) different groups via clustering methods.

\[
T_{\text{ref}}^{(p\% \times m) \times q} = \mathcal{C}_K^{(p\% \times m) \times 1}  = clustering(T_{\text{ref}}^{(p\% \times m) \times 16}), \quad \text{where } q = 1
\]

By the clustering analysis and merging the resulting information into the reference-base matrix, the reference-base matrix can be updated as:
\[
\left[
\begin{array}{cccccc}
X & E & T & s & I_{\text{ref}} & \mathcal{C}_K
\end{array}
\right]_{I_{\text{ref}} = 1}
\in
\mathbb{R}^{(p\% \times m) \times (53 + 1)}
\]


Even though the \(C_K\) is computed from the clustering analysis on taxa composition matrix \(T_{\text{ref}}\),
the underlying environmental conditions(\(E_{\text{ref}}\)) are the actual drivers to lead to the clustering results,
based on the fundamental assumption that "the reference taxa-composition is shaped by the environmental conditions".

\begin{figure}[!h]
    \centering
    \includegraphics[width=0.8\textwidth]{../presentation/figures/p10_clustering_results.png}
    \caption{\textit{An example of hierarchical clustering results on the taxa composition matrix of the references
    with selected clustering number \(K\).}}
    \label{fig:p10_clustering_results}
\end{figure}


\subsection{Construct “ideal” taxa composition ruler of environmental factors - Fit a Discriminant Function}

\begin{figure}[!h]
\centering
\includegraphics[width=0.8\textwidth]{../results/ideas_visualization/constructing_ruler.png}
\caption{\textit{The reference sites are used as training data to construct this 'ideal' taxa composition ruler.
($t^{'}_{j}$ represents the raw taxa composition of site \(j\), it has not been transformed into the cluster label \(\mathcal{C}_K\) yet.)}}
\label{fig:constructing_ruler}
\end{figure}

At this stage, there are constrained taxa composition information - cluster labels \(\mathcal{C}_K\) that can be used as
response variables in training the environmental-taxa composition regression. 
Specifically, a discriminant function can be fitted here:
\[
\mathcal{F_{\text{dis}}} : E_{\text{ref}}^{(p\% \times m) \times 5} \to \mathcal{C}_K^{(p\% \times m) \times 1}
\]

This discriminant function \(\mathcal{F_{\text{dis}}}\) fitted on the reference sites tells us how the environmental variables - \(E\) roughly shape the 
taxa composition by assigning each site to one of the taxa composition groups \(\mathcal{C}_K\).

During the training stage, the reference sites are partitioned into the \(K\) taxa composition groups, 
helping to fix the group positions in taxa composition space with the pristine taxa composition part in 
each group. \textbf{However, it does not mean there is only pristine taxa composition in each cluster.
When human disturbance appears, the pristine taxa composition should be shifted to a new position in the taxa composition space,
which is how the disturbed sites look like in the same taxa composition space.}

Therefore, these reference sites are partitioned (by clustering) into different clusters to play the role of 'ideal' metrics
on a ruler of taxa composition (by discriminant function), this ruler measures the 'ideal' taxa composition structure that a site should have given its environmental conditions.

An imaginable scenario is that, when we use the fitted \(\mathcal{F_{\text{dis}}}\) as a ruler to measure the taxa composition of sites that are affected by human disturbance,
the measured 'ideal' taxa composition is not equal to the truly observed taxa composition. 
And this difference in taxa composition is caused by the human disturbance, which was measured by the sediment contamination assessment in the previous step.

\subsection{Mark the “ideal taxa composition” for disturbed sites - Apply the Discriminant Function}

Given the fitted discriminant function \(\mathcal{F_{\text{dis}}}\), we can classify the rest \(1 - p\%\) of the sites
into the taxa composition groups, where reference sites with similar environmental conditions are already assigned into.

Because the clustering analysis was done on the reference sites, the known information on the disturbed sites should look like:
\[
\left[
\begin{array}{ccccc}
X & E & T & s & I_{\text{ref}}
\end{array}
\right]_{I_{\text{ref}} = 0}
\in
\mathbb{R}^{((1 - p\%) \times m) \times (53)}
\]

After applying the discriminant function on these disturbed sites, 
we can know their environmental-deterministic taxa composition groups,
\(\mathcal{C}_K^{((1 - p\%) \times m) \times 1}\).
It expands the disturbed-base matrix to:
\[\left[
\begin{array}{cccccc}
X & E & T & s & I_{\text{ref}} & \mathcal{\hat C}_K
\end{array}
\right]_{I_{\text{ref}} = 0}
\in
\mathbb{R}^{((1 - p\%) \times m) \times (53 + 1)}
\]

Compare it with the reference-base matrix,
we can see that the sites having the same taxa composition cluster \(\mathcal{C}_K\) are now comparable 
with the control of environmental variables \(E\).

To the \(i\) th site in the matrix:

\[\left[
\begin{array}{cccccc}
X & E & T & s & I_{\text{ref}} & \mathcal{C}_K
\end{array}
\right]_{I_{\text{ref}} = 1}
\in
\mathbb{R}^{(p\% \times m) \times (53 + 1)}
\]

If the site has \(\mathcal{C}_{K_i} = \mathcal{\hat C}_{K_j}\), then the \(i\)-th reference site is comparable with the disturbed site \(j\)-th site in the taxa composition space with the control of environmental conditions.
The difference in their taxa composition, \(\delta T_{i,j}\), is caused by the human disturbance, \(\delta X_{i,j}\), between the two sites.

\[
\mathcal{C}_{K_i} = \mathcal{\hat C}_{K_j} \Rightarrow E_{\text{ref}, i}^{(1 \times 5)} \approx E_{\text{dis}, j}^{(1 \times 5)} \Rightarrow \delta T_{i,j} = \mathcal{F}_{reg}(\delta X_{i,j})
\]

Therefore, the sites within the same taxa composition group will be used to fit the regression model - \(\delta T_{i,j} = \mathcal{F}_{reg, k}(\delta X_{i,j})\), 
and these completed groups can be found through
the merging-dismantle process of the two base matrices.

Merging the reference-base matrix and the disturbed-base matrix:
\[
\text{stack}\left(
\left[
\begin{array}{cccccc}
X & E & T & s & I_{\text{ref}} & \mathcal{C}_K
\end{array}
\right]_{I_{\text{ref}} = 1}
,
\left[
\begin{array}{cccccc}
X & E & T & s & I_{\text{ref}} & \mathcal{\hat C}_K
\end{array}
\right]_{I_{\text{ref}} = 0}
\right)
\]
\[
\Rightarrow
\left[
\begin{array}{cccccc}
X & E & T & s & I_{\text{ref}} & \mathcal{\hat C}_K
\end{array}
\right]
\]

Split the merged matrix into \(K\) submatrices, where each submatrix contains the same cluster label \(\mathcal{C}_k\):

\[
\left[
\begin{array}{cccccc}
X & E & T & s & I_{\text{ref}} & \mathcal{\hat C}_K
\end{array}
\right]
=
\left\{
\begin{array}{ll}
\left[
\begin{array}{cccccc}
X & E & T & s & I_{\text{ref}} & \mathcal{C}_1
\end{array}
\right] & \text{if } \mathcal{C}_K = 1 \\[1.2em]
\left[
\begin{array}{cccccc}
X & E & T & s & I_{\text{ref}} & \mathcal{C}_2
\end{array}
\right] & \text{if } \mathcal{C}_K = 2 \\[0.8em]
\quad\vdots & \quad
\vdots \\[0.8em]
\left[
\begin{array}{cccccc}
X & E & T & s & I_{\text{ref}} & \mathcal{C}_K
\end{array}
\right] & \text{if } \mathcal{C}_K = K
\end{array}
\right.
\]

Within each submatrix,
$
\left[
\begin{array}{cccccc}
X & E & T & s & I_{\text{ref}} & {\mathcal{C}_k}
\end{array}
\right]
$, we will numerically measure the difference in the taxa composition 
between the degraded sites and the reference sites,
$\delta T_k$, this distance in taxa composition will be explained by the 
relative stress level of each site, \(\delta X_k\).

\begin{figure}
\centering
\includegraphics[width=0.8\textwidth]{../presentation/figures/p12_fit_apply_discriminant_function.png}
\caption{\textit{Visualization of the fitting and application of the discriminant function that assigns disturbed sites to the environmentally determined taxa composition groups.
}}
\label{fig:p12_fit_apply_discriminant_function}
\end{figure}

\subsection{Measure the difference from “pristine” to “true” taxa composition - Multivariate Gaussian Deviation Index }

\begin{figure}[!h]
    \centering
    \includegraphics[width=0.8\textwidth]{../results/ideas_visualization/mark_taxa_difference_and_explain.png}
    \caption{\textit{The difference in taxa composition between the observed($t_{j}$) and ruler measured($t^{'}_{j}$) is connected to the sediment contamination level($s$).}}
    \label{fig:mark_taxa_difference_and_explain}
\end{figure}

Within each taxa-composition group $\mathcal{C}_k$, let $\mathcal{R}_k$ denote the set of reference sites ($I_{\text{ref}}=1$) and $\mathcal{D}_k$ the set of disturbed sites ($I_{\text{ref}}=0$).
We construct a site-level deviation metric that quantifies how far a site's observed community is from the pristine expectation of its group while controlling for environmental setting via $\mathcal{C}_k$.

% \paragraph{Choice of scale for community data.}
Because taxa compositions are multivariate and often compositional/zero-inflated, 
we first work on a transformed scale using the Hellinger transformation:
\[
\phi_{\mathrm{Hel}}:\;\mathbb{R}^{16}_{\ge 0}\rightarrow\mathbb{R}^{16}, \quad
\phi_{\mathrm{Hel}}(\mathbf{t})=
\left(
\sqrt{\frac{t^{(1)}}{\sum_{\ell=1}^{16} t^{(\ell)}}},\;
\dots,\;
\sqrt{\frac{t^{(16)}}{\sum_{\ell=1}^{16} t^{(\ell)}}}
\right).
\]
This transformation converts each taxon abundance to the square root of its relative abundance, 
reducing the influence of highly dominant taxa while preserving ecological distance relationships.
All subsequent quantities are computed on $\phi_{\mathrm{Hel}}(T)$; 
to simplify notation we overwrite $T \leftarrow \phi_{\mathrm{Hel}}(T)$.

There are other transformations that may be preferred depending on the context (e.g., log-ratio transforms, or raw counts),
the Hellinger transformation is tentative and can be replaced as needed.


After the transformation, for group $k$, compute the reference centroid and covariance
\[
\boldsymbol{\mu}_k \;=\; \frac{1}{|\mathcal{R}_k|}\sum_{i\in\mathcal{R}_k} T_{i}, 
\qquad
\boldsymbol{\Sigma}_k \;=\; \mathrm{Cov}\{T_{i}: i\in\mathcal{R}_k\} + \lambda I_{16},
\]
where $\lambda>0$ is a small ridge term to ensure invertibility and numerical stability, and 
$I_{16}$ is the $16\times 16$ identity matrix.
These parameters $(\boldsymbol{\mu}_k,\boldsymbol{\Sigma}_k)$ define a 
multivariate Gaussian-like distribution in the $16$-dimensional taxa space, representing the 
\emph{pristine community cloud} for group $k$.
Under this view, each reference site is a draw from 
$\mathcal{N}(\boldsymbol{\mu}_k, \boldsymbol{\Sigma}_k)$, 
and the geometric shape of this cloud is an ellipsoid whose orientation and size are determined by 
$\boldsymbol{\Sigma}_k$.

\subsubsection{Value-based Measurement: Z-score Community Index (ZCI)}
For any site $j$ in group $k$ (reference or disturbed), define the multivariate standardized 
deviation from the pristine centroid as the Mahalanobis distance:
\[
\mathrm{ZCI}_{k,j} \;=\; \sqrt{\big(T_{j}-\boldsymbol{\mu}_k\big)^{\top}\,\boldsymbol{\Sigma}_k^{-1}\,\big(T_{j}-\boldsymbol{\mu}_k\big)}.
\]
This measures how far $T_j$ lies from the center of the pristine Gaussian cloud, in units that 
account for both taxon-specific variability and cross-taxon correlations.
It effectively reduces the $16$-dimensional deviation vector to a \emph{single scalar score} 
while preserving the anisotropic geometry of the reference distribution.

When a diagonal approximation is preferred, use the ``sum of squared z-scores'' variant:
\[
\mathrm{ZCI}^{(\mathrm{diag})}_{k,j} \;=\; \sqrt{\sum_{\ell=1}^{16}\left(\frac{T^{(\ell)}_{j}-\mu^{(\ell)}_{k}}{\sigma^{(\ell)}_{k}}\right)^{2}},
\]
where $\sigma^{(\ell)}_{k}$ is the reference standard deviation of taxon $\ell$ in group $k$ 
(robust alternatives such as median absolute deviation may also be used).
This ignores inter-taxon correlations, treating the pristine cloud as an axis-aligned hypersphere, 
which can be more stable when the number of reference sites is small relative to the number of taxa.

\begin{figure}[!h]
\centering
\includegraphics[width=0.8\textwidth]{../presentation/figures/p15_details_of_taxa_difference_in_1dimention.png}
\caption{\textit{Visualization of the details of taxa community structure differences measured in one-dimensional ZCI.}}
\label{fig:p15_details_of_taxa_difference_in_1dimention}
\end{figure}

\subsubsection{Vector-based Measurement: multi-dimensional ZCI}
The scalar $\mathrm{ZCI}_{k,j}$ summarizes deviation magnitude but discards the 
\emph{direction} of change in community composition.  
To retain more structure, the same Gaussian framework can be used to construct a 
multi-dimensional ZCI:

\begin{enumerate}
    \item \textbf{Whitening of deviations
    \footnote{Whitening means: Centering (subtracting \(\mu_k\)), rescaling and rotating so that the reference
    covariance becomes the identity matrix. Knowing that \(\Sigma_k = \frac{1}{|T|-1} (T - \mu)^T (T - \mu)\),
    replacing the \(T\) with \(\tilde{T} = \Sigma_k^{-1/2} (T - \mu)\),
    the new covariance matrix \(\tilde \Sigma_k\) becomes \(\frac{1}{|\tilde T| - 1} (\tilde{T} - \tilde \mu)^T (\tilde{T} - \tilde \mu) = I \) 
    . Here, \(T\) and \(\mu\) are both matrices and \(\Sigma_k\) is a non-singular matrix.}
    :} For each site $j$ in group $k$, compute the whitened deviation vector
    \[
    \tilde{T}_{k,j} = \boldsymbol{\Sigma}_k^{-1/2} (T_j - \boldsymbol{\mu}_k),
    \]
    where $\boldsymbol{\mu}_k$ and $\boldsymbol{\Sigma}_k$ are estimated from the reference sites.  
    Denote the matrix of whitened deviations for \emph{reference} sites as 
    $\tilde{T}_{k,\mathrm{ref}} \in \mathbb{R}^{n_{\mathrm{ref},k} \times 16}$.  
    In this whitened space, the reference cloud is isotropic and centered at the origin.
    
    \item \textbf{PCA fitted on whitened reference sites:}  
    Perform principal component analysis (PCA) on $\tilde{T}_{k,\mathrm{ref}}$ to obtain the loading matrix $V_k$.  
    Retain the first $d$ principal axes $V_{k,(1:d)}$, where $d=2$ gives a two-dimensional ZCI.

    \item \textbf{PCA applied to disturbed sites:}  
    For each disturbed site $j$, compute its whitened deviation $\tilde{T}_{k,j}$ using the \emph{same} $\boldsymbol{\mu}_k$ and $\boldsymbol{\Sigma}_k^{-1/2}$ from the reference sites, and project it onto the retained principal axes:
    \[
    \big( \mathrm{ZCI}^{(1)}_{k,j}, \mathrm{ZCI}^{(2)}_{k,j} \big)
    = \tilde{T}_{k,j} \; V_{k,(1:2)}.
    \]
\end{enumerate}

These coordinates preserve both magnitude and orientation of deviation in the most informative 
subspace of the pristine community cloud, enabling more nuanced comparisons between sites 
that have similar scalar ZCI values but differ in the \emph{type} of community shift.  
The scalar $\mathrm{ZCI}_{k,j}$ can be recovered as the Euclidean norm of these coordinates.

\begin{figure}[!h]
\centering
\includegraphics[width=0.8\textwidth]{../presentation/figures/p18_all_sites_in_2dimensional.png}
\caption{\textit{Visualization of the details of taxa community structure differences measured in two-dimension ZCI.}}
\label{fig:p18_all_sites_in_2dimensional}
\end{figure}

\subsubsection{Direction, interpretation, and optional 0--100 scaling.}

By construction, smaller values indicate communities closer to the pristine expectation for their environment; larger values indicate stronger deviation (putative impact).
For reporting, we optionally map ZCI to a condition scale where larger is better:
\[
\mathrm{ZCI}^{\star}_{k,j} \;=\; 100\,\big(1-\widehat{F}_k(\mathrm{ZCI}_{k,j})\big),
\]
with $\widehat{F}_k$ the empirical CDF of $\mathrm{ZCI}$ computed from \emph{reference} sites in group $k$. 
Under this calibration, reference sites cluster near higher scores (closer to $100$), while progressively disturbed sites trend toward $0$.

\subsection{Principal Coordinates of Neighbour Matrices (PCNM) for spatial eigenvectors}

To adjust the ZCI--stress relationships for residual spatial structure,
we derive spatial eigenvectors (PCNM variables) that capture multi-scale spatial autocorrelation
among the same set of $m$ sites in a specific cluster. 
These act as candidate covariates prior to fitting the piecewise quantile regression model.

Let $(x_j,y_j)$ be the planar (or projected) coordinates for site $j=1,\dots,m$.
Let $\mathbf{1}$ be an $m$-vector of ones, $\mathbf{I}_m$ the $m\times m$ identity, 
and the centering matrix $\mathbf{J}=\mathbf{I}_m-\tfrac{1}{m}\mathbf{1}\mathbf{1}^T$.

To derive spatial eigenvectors, we first compute Euclidean (or hydrologic, if river network) 
distances to form the $m\times m$ distance matrix $\mathbf{D}$ with entries:
\[
D_{ij}=\sqrt{(x_i-x_j)^2+(y_i-y_j)^2}
\]

Next, we choose a connectivity threshold $d_0$ as the maximum edge length of the minimum spanning tree to ensure a connected neighbour graph. Alternative approaches include using the maximum nearest-neighbour distance.

We then construct a truncated distance matrix $\mathbf{T}$ by defining 
\[
T_{ij} = \begin{cases}
D_{ij} & \text{if } D_{ij} \le d_0 \\
4d_0 & \text{otherwise}
\end{cases}
\]

where the large constant enforces separation of non-neighbours. 

The PCoA transform is applied by forming $\mathbf{A} = -\tfrac{1}{2}\,\mathbf{J}\,\mathbf{T}^{\circ 2}\,\mathbf{J}$ (where $^{\circ 2}$ denotes elementwise square), followed by eigen-decomposition $\mathbf{A}=\mathbf{V}\boldsymbol{\Lambda}\mathbf{V}^T$ with eigenvalues $\lambda_1\ge\cdots\ge\lambda_m$ and eigenvectors $\mathbf{v}_k$.
\[
\mathbf{A} = -\tfrac{1}{2}\,\mathbf{J}\,\mathbf{T}^{\circ 2}\,\mathbf{J} = \mathbf{V}\boldsymbol{\Lambda}\mathbf{V}^T, \quad
\]

We retain eigenvectors with positive eigenvalues $\lambda_k>0$ and 
scale them as $\mathbf{s}_k=\mathbf{v}_k\sqrt{\lambda_k}$. 
These orthogonal PCNM vectors span spatial patterns from broad (large $\lambda_k$) 
to fine scales. For screening, we compute Moran's $I$ for each $\mathbf{s}_k$ 
using a binary (or inverse-distance) weight matrix based on $d_0$, 
retaining only spatially autocorrelated vectors using FDR or adjusted $p$-values.
 Forward selection based on AIC or adjusted $R^2$ can be applied to avoid overfitting.

Finally, we collect the retained spatial vectors in $\mathbf{S}_{\text{sel}}\in\mathbb{R}^{m\times q_s}$ and add these to subsequent ZCI quantile regression models (using the same $\mathbf{S}_{\text{sel}}$ across $\tau$ for comparability). We test residual Moran's $I$ and iterate if needed.

% The retained PCNM set provides an orthogonal spatial basis approximating latent spatial processes not captured by measured environmental variables. Incorporating $\mathbf{S}_{\text{sel}}$ isolates the anthropogenic (contamination) signal in ZCI–stress relationships, reducing bias and inflated Type I error from spatial autocorrelation. Resulting variation partitioning can then attribute explained variance uniquely to contamination versus spatial structure.




\subsection{Build ZCI indicator of sediment contamination levels – Piecewise Quantile Regression Model}

The $\mathrm{ZCI}$ score reflects the degree of deviation in taxa composition from the pristine expectation within each group $k$, given the same environmental context. 
Because both the stress level and the community condition are influenced by a range of measured and unmeasured factors, it is reasonable to model the conditional distribution of the stress level \emph{given} the community departure.  
This regression is used only as a statistical association to infer likely stress levels from observed $\mathrm{ZCI}$ values and does \textbf{not} imply a causal relationship between stress and community departure.

Within each group \(k\), we relate the relative stress level to the ZCI deviation \emph{and} the selected spatial eigenvector predictors derived earlier. Let
\[
z_{k,j}:=\mathrm{ZCI}_{k,j}, \qquad \mathbf{s}_{k,j}\in\mathbb{R}^{q_s}\text{ be row }j\text{ of }\mathbf{S}_{\text{sel}} (q_s \text{ retained PCNM vectors}).
\]
We model
\[
\delta X_{k,j} = \mathcal{F}_{k}\big(z_{k,j},\,\mathbf{s}_{k,j}\big)+\varepsilon_{k,j},
\]
where $\delta X_{k,j}$ is the relative stress (e.g., $s_j-\mathrm{median}\{s_i:i\in\mathcal{R}_k\}$). Using the \emph{same} spatial basis $\mathbf{S}_{\text{sel}}$ across all quantiles $\tau$ preserves comparability of slope and breakpoint inference by ensuring spatial adjustment does not vary with $\tau$.

Given potential nonlinearity in $z$ and heteroscedasticity, we use a piecewise linear quantile formulation in $z$ while treating spatial terms additively:
\[
Q_{\delta X\mid Z,S}^{(k)}(\tau \mid z, \mathbf{s}) = f_{k,\tau}(z,\mathbf{s}) = \beta_{0,\tau}^{(k)} + \beta_{1,\tau}^{(k)} z + \sum_{m=1}^{M} \gamma_{m,\tau}^{(k)} (z-\kappa_m)_+ + \sum_{r=1}^{q_s} \alpha_{r,\tau}^{(k)} s^{(r)},
\]
with breakpoints $\kappa_1<\cdots<\kappa_M$ placed on the ZCI axis only (spatial covariates are not segmented). Parameters solve the check-loss minimization
\[
\widehat{\boldsymbol{\theta}}_{\tau}^{(k)} \in \arg\min_{\boldsymbol{\theta}} \sum_{j\in\mathcal{C}_k} \rho_{\tau}\Big( \delta X_{k,j} - f_{k,\tau}(z_{k,j},\mathbf{s}_{k,j}) \Big), \qquad \rho_{\tau}(u)=u\{\tau-\mathbf{1}(u<0)\}.
\]
The fitted conditional quantile surface is then
\[
\widehat{Q}_{\delta X\mid Z,S}^{(k)}(\tau \mid z,\mathbf{s}) = \widehat{\beta}_{0,\tau}^{(k)} + \widehat{\beta}_{1,\tau}^{(k)} z + \sum_{m=1}^{M} \widehat{\gamma}_{m,\tau}^{(k)} (z-\kappa_m)_+ + \sum_{r=1}^{q_s} \widehat{\alpha}_{r,\tau}^{(k)} s^{(r)}.
\]
Including the spatial term ensures that breakpoint and slope interpretations are attributed to contamination-driven community departure rather than residual spatial patterning.

\subsubsection{Hypothesis testing for degradation -- Quantile-based threshold inference}

In many applications, a binary classification of a site as ``degraded'' or ``non-degraded'' 
is more actionable than estimating its exact stress level. 
This decision problem can be formulated as a one-sided hypothesis test, conditioning on the 
observed community departure (ZCI):

\begin{itemize}
    \item \textbf{Step 1 -- Define degradation threshold.}  
    For each group $k$, choose a stress threshold $x_{k}^*$ 
    (e.g., a regulatory limit or an ecologically relevant benchmark) 
    that separates degraded from non-degraded sites.

    \item \textbf{Step 2 -- Predict conditional stress distribution.}  
    For a site $j$ with observed $\mathrm{ZCI}_{k,j}=z$, 
    use the fitted quantile regression model 
    ${Q}_{\delta X\mid Z,S}(\tau | z, \mathbf{s})$ over a grid of quantile levels 
    $\tau \in (0,1)$ to approximate the conditional distribution 
    $F_{\delta X\mid Z, S}^{(k)}(x | z, \mathbf{s})$. 
    This is done by inverting the quantile function across $\tau$.

    \item \textbf{Step 3 -- Compute $p$-value for degradation.}  
    The hypothesis test is:
    \[
    H_0: \delta X_{k,j} \le x_{k}^* \quad \text{vs.} \quad H_a: \delta X_{k,j} > x_{k}^* .
    \]
    The conditional $p$-value is then
    \[
    p = 1 - F_{\delta X\mid Z,S}^{(k)}\!\left( x_{k}^* \,\middle|\, z, \mathbf{s} \right),
    \]
    which represents the probability, given the site's ZCI and spatial context, that the stress level 
    exceeds the degradation threshold.
\end{itemize}

If $p$ is below a chosen significance level $\alpha$ (e.g., $0.05$), 
the site is classified as degraded; otherwise, it is classified as non-degraded.  
This approach converts the regression output into a probabilistic decision rule while 
controlling for environmental setting via the group-specific model.

\begin{figure}[!h]
\centering
\includegraphics[width=0.9\textwidth]{../presentation/figures/p16_degraded_threshold_and_quantile_regression.png}
\caption{\textit{A pre-fixed degradation threshold on the conditional stress level distribution and the correspondingly predicted quantile value \(\hat s_{\tau}\)}}
\label{fig:p16_degraded_threshold_and_quantile_regression}
\end{figure}


\subsection{Indicator power and robustness with respect to sample size (tentative)}
Power and robustness evaluation of the indicator can be divided into the following specific aims: 
(i) quantify precision of slopes and breakpoint effects and conditional quantiles as sample size varies 
and (ii) estimate power for detecting contamination structure and degradation thresholds, all \emph{within each group $k$}.

	\textbf{Targets:} slopes $\beta_{1,\tau}^{(k)}$, changes $\gamma_{m,\tau}^{(k)}$, breakpoint reliability, pseudo-$R^2$, degradation test (Type I / power), and conditional quantiles $\widehat Q_{\delta X|Z,S}$.

	\textbf{Procedure (tentatively designed steps):}
\begin{enumerate}
    \item \emph{Baseline fit:} Fit full piecewise quantile model (fixed $\kappa_m$, fixed $S_{\text{sel}}$) for $\tau \in \mathcal{T}$; store parameters and residuals; test residual Moran's $I$.
    \item \emph{Bootstrap (uncertainty):} If no residual spatial autocorrelation: site bootstrap; else block bootstrap. Refit using original $S_{\text{sel}}$; derive percentile CIs and relative widths.
    \item \emph{Subsampling (precision curves):} For a grid of reduced sizes $n_k^{(g)}$, draw $R$ subsamples preserving 
    ratio of \(\frac{reference}{disturbed}\); refit; summarize bias and RMSE of slopes / predicted quantiles; locate diminishing returns size.
    % \item \emph{Effective sample size:} When mild spatial correlation persists, report $ n_{k,\text{eff}} = \dfrac{n_k}{1 + (n_k-1)\bar{\rho}_k }$ and inflate SEs by $\sqrt{n_k/n_{k,\text{eff}}}$.
    \item \emph{Power simulation:} \(H_0\): $\beta_{1,\tau}=\gamma_{m,\tau}=0$; \(H_1\): baseline (and reduced effect). Simulate $L$ datasets holding $(z, S_{\text{sel}})$ fixed; estimate power for joint slope/breakpoint test and degradation classification at threshold $x_k^*$.
    \item \emph{Breakpoint reliability:} Accept $\kappa_m$ if CI width $< 0.3$ of ZCI span \emph{and} $\Pr(\gamma_{m,\tau}\neq 0 \text{ for some } \tau) \ge 0.8$; otherwise reduce segments or increase sample size.
    \item \emph{Outputs:} (a) precision and power curves vs. $n_k$; (b) recommended minimum $n_k$ meeting power $\ge 0.8$ (median quantile) and reliability rule; (c) table of median (IQR) parameter estimates.
\end{enumerate}

	\textbf{Interpretation:} Early plateau in precision implies reallocating effort to under-sampled groups or spatial gaps.
     If slopes are consistent but breakpoints unreliable, report continuous gradients instead of thresholds.
    Reliable, sharp breakpoints support categorical management triggers.


\section{Preliminary exploration}

In this section, I implemented a simplified completed workflow based 
on the methodology described in Section Methodology. 
It explored the practical application of the proposed method to support the later composite workflow.
Specifically, the major preliminary steps are as follows:

\begin{enumerate}

\item \textbf{Collect comparable data.}

There are three available datasets, containing the three raw data types: zoobenthic community data (311×16), chemical data (104×30), and environmental data (289×7).
The three shared the same identical index - StationID, which supports data merging to prepare completed data for 
each site. After merging there is a combined dataset with 104 rows and 53 columns, containing all three types of data.

\textbf{Key Results:} \(\left[
\begin{array}{ccc}
X & E & T
\end{array}
\right]
\in
\mathbb{R}^{m \times (30 + 5 + 16)}\) was prepared
\footnote{
The column numbers were not consistent(\(51 \neq 53\))
because StationID and location information were not used in the analysis.
Later, the location information can be included to support spatial analysis.
}.

\item \textbf{Assess sediment contamination and Identify reference and degraded sites.}

A log-transformation was applied to the chemical data to reduce dominance by high-value variables.
Then PCA was performed on the transformed chemical data and several principal components were selected to explain
the major variation of pollutant elements. By simply standardizing and summing these PCs, comprehensive stress values were computed for each site.
To keep consistency with Jian's analysis, I used the name "SumReal" to refer these stress values (levels).
Current statistics results shown "the higher are the stress scores, the less are the pollutant elements concentrations",
but this will be further explored in the future.

Based on the computed stress scores,  \(p\%\) was temporarily set to 20\% to identify reference sites and the degraded sites were symmetrically defined.
Out of assumption, there were no or minimal human disturbances on the reference sites, their taxa composition was shaped by environmental conditions only.


\textbf{Key Results:} \(\left[
\begin{array}{ccccc}
X & E & T & s & I_{ref}
\end{array}
\right]
\in
\mathbb{R}^{m \times (51 + 2)}\), \(\) was prepared

\item \textbf{Cluster reference sites by taxa community composition.}

Turn to the zoobenthic community data of these references, IQR method was applied to detect outliers
and then octave transformation was applied to reduce extreme values' impact, 
considering that taxa in low abundance do not mean they are not important.

Then clustering was applied to identify major taxa community patterns across different environmental conditions,
and \(K\) clusters were identified. Here \(K\) was set to 3 through the hierarchical clustering method.
These sites assigned into each \(k\) cluster represent "ideal" taxa structures,
each totally shaped by the range of environmental conditions of the corresponding cluster.

\textbf{Key Results:} 
\(\mathcal{C}_K\)
and
\(\left[
\begin{array}{cccccc}
X & E & T & s & I_{ref} & \mathcal{C}_K
\end{array}
\right]_{I_{ref} = 1}
\in
\mathbb{R}^{(p\% \times m) \times (53 + 1)}\), were prepared.

\item \textbf{Fit Discriminant Function of environmental factors for taxa clusters.}

Based on the identified clusters of these reference sites,
a discriminant function was fitted to predict the taxa cluster membership of each site based on its environmental variables.
\footnote{It is both acceptable to say "environmental clusters" or "taxa clusters", owing to the assumption that 
these reference sites have their taxa composition totally shaped by the environmental conditions.} 

\textbf{Key Results:} 
\(\mathcal{F}_{dis}: e^{(1, 5)} \to \hat C_K\) was fitted.

\item \textbf{Apply the Discriminant Function to rest disturbed sites to group them}

To the (1 - \(p\%\)) disturbed sites, including the degraded sites, 
\(\mathcal{F}_{dis}\) was applied to assign each site to one of the identified taxa clusters.

\textbf{Key Results:} 
\(\mathcal{F}_{dis}: e^{(1, 5)} \to \hat C_K\) was applied, 
\(\left[
\begin{array}{cccccc}
X & E & T & s & I_{ref} & \mathcal{\hat C}_K
\end{array}
\right]_{I_{ref} = 0}\) was prepared.

\item \textbf{Construct endpoints and compute ZCI in each cluster.}

Within each cluster, the reference sites and degraded sites were used to construct endpoints, 
unlike the Multivariate Gaussian cloud of reference sites, 
the endpoints were simply computed as the means of 3 selected references and 3 degraded sites.
The endpoints were used to numerically scale the taxa community structure and compute the Zoobenthic Condition Index via 
Bray-Curtis ordination method. The \(ZCI_{k}\) reflected the distance of the disturbed sites to the endpoints in cluster \(k\).

\textbf{Key Results:} 
\(ZCI_{k}, (k \in {1, ..., K})\) was computed; \(ZCI_{k, j}\) reflects the distance of site \(j\) to its cluster \(k\) endpoints.

\item \textbf{Evaluate the ZCI vs SumReal relationship by quantile regression.}

On top of the \(ZCI\) and stress level \(s\), a piecewise quantile regression was fitted to evaluate their relationship across 
the three taxa clusters. The breakpoints were found by grid search method with a pre-defined searching range, which 
was set manually based on the preliminary exploration.

\textbf{Key Results:} 
\(f_{k, \tau}(z), (k \in {1, ..., K})\) was fitted; \(\hat \theta_{\tau}^{(k)}\) was solved for each cluster \(k\).

\item \textbf{Preliminary exploration of spatial structure with simulation}

An isolated PCNM (spatial eigenfunction) simulation was run, separate from previous workflow steps, to illustrate extraction of spatial structure across increasing complexity. Three scenarios were simulated (low to higher complexity): (i) equally spaced 1D sites, (ii) equally spaced 2D lattice, and (iii) irregular 2D sites with two clusters. The first two yield regular distance matrices and smoothly ordered (broad- to fine-scale) eigenvectors; the clustered irregular layout produces wave-like spatial patterns in the leading eigenfunctions.

\textbf{Key Results:} \(PCNM: D \to S_{\text{sel}}\) was carried out; \(S_{\text{sel}}\) comprises the Moran's I–screened spatial eigenvectors derived from the truncated neighbour distance matrix \(D\).

\item \textbf{Power and sensitivity analysis on quantile regression with simulation}

A deterministic piecewise quantile regression model with fixed breakpoint \(\kappa\) and heteroscedastic noise was used to generate synthetic data. Subsamples of sizes \(n=20,50,80,\dots,500\) were drawn to assess estimation precision of key parameters (\(\hat y_{\tau}\), \(\beta_{1,\tau}\), \(\delta_{\tau}\)). Bootstrap 95\% CIs and precision curves (RMSE) were computed. Power analysis for detecting non-zero hinge effects (\(\delta\neq 0\)) was conducted via repeated simulation under alternative and null conditions.

\textbf{Key Results:} Preliminary patterns show (i) diminishing RMSE gains beyond moderate \(n\); (ii) wider CIs and reduced power at extreme quantiles; (iii) hinge (\(\delta_{\tau}\)) estimates stabilising more slowly than primary slope estimates. Several implementation and interpretation issues (e.g., behaviour at high \(\tau\) with small \(n\)) were noted for later, more detailed exploration.


\end{enumerate}

\input{sections/subsectionofPreliminaryResults.tex}
\clearpage
\section{Practical Implementation Plan}

This section outlines the specific implementation plan for the thesis project,
detailing the project structure, work patterns and the benefits of this approach.

The current project structure is designed into 9 major folders,
each with a specific storage function as briefly described in
Table \textcolor{blue}{\ref{tab:project_structure}}.

\begin{table}[!h]
\centering
\caption{Project folder structure and description}
\label{tab:project_structure}
\renewcommand{\arraystretch}{1.3}
\begin{tabular}{p{3cm}p{10cm}}
\hline
\textbf{Folder} & \textbf{Description} \\
\hline
\texttt{src/} & Core source code package (\texttt{ecoindex}) with reusable analysis modules. \\
\texttt{notebooks/} & Interactive Jupyter notebooks for prototyping and exploratory work that majorly achieves by modules from \textit{src}. \\
\texttt{tests/} & Unit tests for packages wrote in \texttt{ecoindex}, using \texttt{pytest} to ensure correctness and reliability. \\
\texttt{configs/} & Centralized configuration (random seeds, parameters, file paths),
collective containers for important parameters. \\
\texttt{data/} & Raw and processed datasets, organized for reproducibility. \\
\texttt{artifacts/} & Outputs not easily reproducible, e.g., trained models or serialized results. \\
\texttt{figures/} & Generated plots, charts, and visualizations for reporting. \\
\texttt{reference/} & External references, guidelines, and supporting documentation. \\
\texttt{documents/} & Thesis drafts, LaTeX files, lying close to results folders to keep
synchronous with the results.\\
\hline
\end{tabular}
\end{table}

The following two subsections will elaborate on the principles and rationales behind this project structure design,
and use a specific example to demonstrate the codebase design and working pattern.

\subsection{Research Professional Codebases}
To make a good research professional codebase, I want to follow the principles of modularity, reusability and maintainability.
This means organizing code into reusable components, documenting functionality clearly, 
doing tests for all components, and ensuring that the codebase can be easily understood and modified.

Out of these principles, the code structure and organization become paramount in achieving the work.
It has to be admitted that such structure would cost more time in setting up and analysing 
the meta data, significant amount of time when compared to directly writing the code in a single 
file. However, with the progress of the project, the benefits of this structure will become apparent
and obviously accelerate the research process, which is visible and achievable to
a non-computer science student with the support of these Generative AI tools.

The following example shows a quick starting point for the codebase, where I wrote
and wrote some functions to achieve a given data transformation operation, wrapping
it in a clear structured way into original data frame and testing the correctness 
of these operations. 

\subsubsection{Modularized Function implementation and lightweight application}

As it was mentioned in the methodology part, I wanted to continuously add computed 
results, which were sitewise specific, into the original data frame meanwhile
keeping the whole data frame tiny and efficient. Assuming I had the three raw data sets in my
\textit{data/raw} folder, and I wanted to do a hellinger transformation on the
taxa data set and export the transformed data into the \textit{data/processed} folder.
To achieve this complete operation, I would need the following modules and functions within them:

\begin{table}[!h]
\centering
\caption{Temporary Code Structure for Hellinger Transformation and Integrity of Data Frame}
\label{tab:module_structure}
\renewcommand{\arraystretch}{1.3}
\begin{tabular}{p{3.5cm}p{9.5cm}}
\hline
\textbf{Module} & \textbf{Description} \\
\hline
\textit{src/config.py} & Configuration management for centralized parameter storage and project settings,
like file paths and processing options. \\
\hline
\textit{src/cleaning.py} & Data cleaning utilities for handling missing values, outliers, and data quality issues. \\
\hline
\textit{src/ingest.py} & Data ingestion functions for loading raw datasets from various sources and formats. \\
\hline
% \textit{src/data-io.py} & Input/output operations for reading and writing data files in different formats. \\
% \hline
\textit{src/dataframe-ops.py} & DataFrame manipulation utilities for wrapping new information into layered data frames \\
\hline
\textit{src/transform.py} & Data transformation functions including ecological transformations like Hellinger. \\
\hline
\textit{src/pipeline.py} & Pipeline orchestration for chaining multiple data processing steps together. \\
\hline
\textit{notebooks/0-interim-data.py} & Notebook to invoke the above functions in practical scenarios,
lightweight code snippets for quick testing and validation. \\
\hline
\textit{tests/test-*.py} & Unit tests for validating the functionality of individual modules and functions
in above modules(\textit{src/*}). \\
\hline
\end{tabular}
\end{table}

The whole modules are not only designed for Hellinger Transformation, but it is a good example and there will be more
functions and module added in this structure to make the project concise and reproducible.

\begin{figure}[!h]
    \centering
    \includegraphics[width=0.9\textwidth]{../results/preliminary_results/implementPlan_example_codebase.png}
    \caption{Demonstration of Workflow for Implementing and Integrating Hellinger Transformation into Complete Data Processing Pipeline}
    \label{fig:implementPlan_example_codebase}
\end{figure}

With this highly integrated modularization, the previously long and complex code can be enclosed in these
separated modules and the application work can be achieved by a few lines of code.
Like the following lines of code in Figure \textcolor{blue}{\ref{fig:example_lightweight_code_completed}} achieve the whole transformation and data merging work
in the Jupyter notebook, which used to require much more extensive and vulnerable code within the same notebook:

\begin{figure}[!h]
    \centering
    \includegraphics[width=0.9\textwidth]{../results/preliminary_results/example_lightweight_code_completed.png}
    \caption{The lines of code needed to achieve the above discussed operations in practical scenarios and the resulting integrated data frame from the operation}
    \label{fig:example_lightweight_code_completed}
\end{figure}




\subsection{General Project Structure and Rationale}

Generally speaking, the project follows a modular and well-organized structure to support both 
research reproducibility and software engineering best practices.
The folder layout is as follows:

\begin{itemize}
    \item \textbf{configs}: centralizes configuration files (e.g., paths, random seeds, parameter settings), ensuring experiments are reproducible and adjustable without modifying code.
    \item \textbf{data}: stores raw, interim, and processed datasets in a structured hierarchy, making data provenance transparent and simplifying pipeline automation.
    \item \textbf{documents}: contains draft texts, LaTeX notes, and thesis-related writeups, providing a bridge between research outputs and manuscript preparation.
    \item \textbf{figures}: keeps all generated visualizations, plots, and diagrams organized for easy reuse in the thesis and publications.
    \item \textbf{notebooks}: hosts exploratory Jupyter notebooks, which serve as an interface for practical implementation, quick testing, and visualization.
    \item \textbf{reference}: used for bibliographic resources, papers, and supplementary literature, ensuring traceability of scientific background.
    \item \textbf{src}: holds the main source code in a modular format (e.g., \texttt{ingest.py}, \texttt{cleaning.py}, \texttt{pipeline.py}), which separates concerns between ingestion, cleaning, transformation, and higher-level operations.
    \item \textbf{tests}: includes unit tests and validation scripts to ensure correctness and robustness of each module across different scenarios.
    \item \textbf{artifacts}: stores intermediate results, logs, and model checkpoints, preserving computational outcomes for reproducibility and debugging.
    \item \textbf{pyproject.toml}: defines the project environment, dependencies, and metadata, which standardizes reproducibility across systems.
\end{itemize}

\noindent This design provides several benefits:
\begin{enumerate}
    \item \textbf{Reproducibility}: Configurations, raw data, and processed results are explicitly separated, making workflows transparent and repeatable.
    \item \textbf{Scalability}: Modular code in \texttt{src/} and standardized data storage enable extensions (e.g., adding new datasets or models) with minimal disruption.
    \item \textbf{Clarity}: Clear distinction between code, data, results, and documentation reduces confusion and facilitates collaboration or future reuse.
    \item \textbf{Professionalism}: The structure aligns with common practices in industrial data science and academic research, ensuring maintainability and credibility of the project.
\end{enumerate}


\clearpage
\section{Supervisory Dissolution}
	The student agrees that supervision will be dissolved if any of the following happen:
	\begin{itemize}
	\item Two consecutive progress reports are unacceptable
	\item Three consecutive progress reports are concerning/unacceptable
	\item An academic integrity violation is suspected by the supervisor and suspected by at least one other faculty member.
	\end{itemize}
	
\clearpage
\section{Timeline(temporary)}

The below months are in 2025, each enumerated body in each month is the 
1st, 2nd, 3rd and 4th week of that month.

\begin{itemize}
    \item August
        \begin{enumerate}
            \item Make Preliminary Analysis on the raw data, explore the achievability of the general framework.
            \item Make adjustments on the previous proposal based on the validated preliminary results.
            \item Finish the proposal, check and submit the proposal.
            \item Prepare and study the relevant theoretical foundations, about the math, statistical and coding tools:
            for example: Discriminant Function and Ordination method and Project management.
        \end{enumerate}
    \item September
        \begin{enumerate}
            \item To finish and test the \textbf{\textit{data cleaning, operation and transformation modules}}, 
            design and prepare the \textbf{\textit{PCA sediment contamination module}}.
            \item To finish and test the \textbf{\textit{PCA sediment contamination module}},
            design and prepare the \textbf{\textit{Cluster Analysis module}}.
            \item To finish and test the \textbf{\textit{Cluster Analysis module}},
            design and prepare the \textbf{\textit{Discriminant Function Analysis module}}.
            \item To finish and test the \textbf{\textit{Discriminant Function Analysis module}},
            design and prepare the \textbf{\textit{Ordination-ZCI constructor module}}. 
        \end{enumerate}
    \item October
        \begin{enumerate}
            \item To finish and test the \textbf{\textit{Ordination-ZCI constructor module}},
            design and prepare the \textbf{\textit{Piecewise Quantile Regression module}}.
            \item To finish and test the \textbf{\textit{Piecewise Quantile Regression module}},
            design and prepare the \textbf{\textit{Degradation Detection module}}.
            \item To finish and test the \textbf{\textit{Degradation Detection module}},
            organize the whole basic framework.
            \item Prepare and practice the Synthetic Data Method for the project,
            design the \textbf{\textit{Synthetic Data module}}. 
        \end{enumerate}
\end{itemize}

\noindent\rule{\textwidth}{0.4pt}
\centerline{\textbf{Complete basic framework at the finish of Degradation Detection Module}}
\noindent\rule{\textwidth}{0.4pt}

\begin{itemize}
    \item November
        \begin{enumerate}
            \item To finish and test the \textbf{\textit{Synthetic Data module}},
            figure out where to integrate it in the overall framework and
            how to ensure its compatibility with existing modules and test its performance.
            \item Prepare and practice the spatial analysis into the overall framework,
            study and practice deep learning methods, theoretically integrate them into the framework.
            \item Design, finish and test the \textbf{\textit{geographical weighted PCA module}},
            integrate into the established framework and interpret its results against the classic PCA results.
            \item Create new modules or extend existing modules to support spatial analysis and test them.
        \end{enumerate}
    \item December
        \begin{enumerate}
            \item Introduce deep learning methods into the existing framework,
            figure out how to integrate them into specific steps of modules and how to
            validate their performance and interpret their results.
            \item Integrate and adjust the whole modules, ensuring smooth interaction and data flow between them,
            like a software test suite.
            \item Finish 80\% of the first draft.
            \item Include the Satellite-derived data about wildfire into the project,
            explore the spatial analysis combined piecewise quantile regression using the
            established modules.
        \end{enumerate}
\end{itemize}

The below months are in 2026, each enumerated body in each month is the 
1st, 2nd, 3rd and 4th week of that month, '.' symbol means still to be determined.

\begin{itemize}
    \item January
        \begin{enumerate}
            \item Analyse the Satellite-derived data about wildfire and its relevance to
            the established methods, identifying potential improvements and adjustments.
            \item Finish the complete first draft.
            \item Make necessary revisions based on feedback and testing results.
            \item Conduct thorough testing and validation of the entire framework.
        \end{enumerate}
\end{itemize}

\noindent\rule{\textwidth}{0.4pt}
\centerline{\textbf{Complete First draft of the thesis}}
\noindent\rule{\textwidth}{0.4pt}

\begin{itemize}
    \item February
        \begin{enumerate}
            \item Finish the second draft of the thesis.
            \item .
            \item .
            \item Finish the complete thesis(third version).
        \end{enumerate}

    \item March
        \begin{enumerate}
            \item Preparing presentation materials and slides.
            \item Rehearsing the presentation and addressing potential questions.
            \item .
            \item .
        \end{enumerate}
\end{itemize}

\noindent\rule{\textwidth}{0.4pt}
\centerline{\textbf{Complete Second and final draft of the thesis}}
\noindent\rule{\textwidth}{0.4pt}


\begin{itemize}
    \item April
        \begin{enumerate}
            \item Presentation and defence of the thesis, determined by the coordinator.
            \item .
            \item .
            \item .
        \end{enumerate}
\end{itemize}

\noindent\rule{\textwidth}{0.4pt}
\centerline{\textbf{Presentation and defence}}
\noindent\rule{\textwidth}{0.4pt}






\clearpage
\bibliographystyle{plain}  % or use unsrt if you want citations in order of appearance
\bibliography{sections/Reference}   % your .bib file, without .bib extension

\clearpage
\section{Appendix}

\subsection{Tables}

% statistical description for the chemical data within reference, intermediate, and degraded sites based on PCA assessment
% Table 1: Major Metals
\begin{table}[ht]
\centering
\caption{Descriptive Statistics of Major Metals by Site Label}
\label{tab:chem_desc_metals}
\begin{tabular}{>{\centering\arraybackslash}m{2.5cm} >{\centering\arraybackslash}m{1.5cm} >{\centering\arraybackslash}m{2cm} >{\centering\arraybackslash}m{2cm} >{\centering\arraybackslash}m{2cm}}
\toprule
 & \textbf{SumReal} & \textbf{degraded} & \textbf{intermediate} & \textbf{reference} \\
\midrule
\multirow[t]{2}{*}{Al} & mean & 4276.423 & 6380.140 & 4319.381 \\
 & std & 2888.769 & 5523.949 & 1767.861 \\
\cline{1-5}
\multirow[t]{2}{*}{As} & mean & 2.186 & 1.777 & 2.232 \\
 & std & 1.602 & 1.290 & 1.041 \\
\cline{1-5}
\multirow[t]{2}{*}{Bi} & mean & 17.085 & 17.505 & 17.622 \\
 & std & 10.352 & 10.273 & 9.722 \\
\cline{1-5}
\multirow[t]{2}{*}{Ca} & mean & 28180.500 & 33518.930 & 28480.714 \\
 & std & 14031.433 & 11400.266 & 11870.107 \\
\cline{1-5}
\multirow[t]{2}{*}{Cd} & mean & 0.535 & 0.351 & 0.271 \\
 & std & 0.649 & 0.202 & 0.233 \\
\cline{1-5}
\multirow[t]{2}{*}{Co} & mean & 4.049 & 4.497 & 3.984 \\
 & std & 1.733 & 2.209 & 1.118 \\
\cline{1-5}
\multirow[t]{2}{*}{Cr} & mean & 13.254 & 12.830 & 9.007 \\
 & std & 16.373 & 11.835 & 2.937 \\
\cline{1-5}
\multirow[t]{2}{*}{Cu} & mean & 16.958 & 18.082 & 12.946 \\
 & std & 22.388 & 29.120 & 9.003 \\
\cline{1-5}
\multirow[t]{2}{*}{Fe} & mean & 9495.000 & 11246.789 & 9650.905 \\
 & std & 5392.824 & 6804.654 & 3856.739 \\
\cline{1-5}
\multirow[t]{2}{*}{Hg} & mean & 0.474 & 0.324 & 0.196 \\
 & std & 1.230 & 0.420 & 0.365 \\
\cline{1-5}
\multirow[t]{2}{*}{K} & mean & 818.927 & 1285.558 & 845.657 \\
 & std & 638.053 & 1092.550 & 411.332 \\
\cline{1-5}
\multirow[t]{2}{*}{Mg} & mean & 12849.500 & 15204.175 & 12269.143 \\
 & std & 6104.202 & 5764.037 & 5281.794 \\
\cline{1-5}
\multirow[t]{2}{*}{Mn} & mean & 161.228 & 188.900 & 161.905 \\
 & std & 76.973 & 86.663 & 57.883 \\
\cline{1-5}
\multirow[t]{2}{*}{Na} & mean & 118.998 & 134.042 & 123.611 \\
 & std & 49.081 & 43.693 & 41.021 \\
\cline{1-5}
\multirow[t]{2}{*}{Ni} & mean & 11.225 & 12.399 & 9.136 \\
 & std & 8.851 & 8.424 & 3.542 \\
\cline{1-5}
\multirow[t]{2}{*}{Pb} & mean & 12.515 & 8.774 & 8.573 \\
 & std & 32.312 & 22.204 & 18.750 \\
\cline{1-5}
\multirow[t]{2}{*}{Sb} & mean & 17.262 & 16.765 & 18.001 \\
 & std & 11.879 & 13.115 & 13.743 \\
\cline{1-5}
\multirow[t]{2}{*}{V} & mean & 15.274 & 18.353 & 15.183 \\
 & std & 7.012 & 9.560 & 4.408 \\
\cline{1-5}
\multirow[t]{2}{*}{Zn} & mean & 52.732 & 46.181 & 35.677 \\
 & std & 48.896 & 44.586 & 17.938 \\
\cline{1-5}
\bottomrule
\end{tabular}
\end{table}

% Table 2: Organic Carbon and Chlorinated Benzenes
\begin{table}[ht]
\centering
\caption{Descriptive Statistics of Organic Carbon and Chlorinated Benzenes}
\label{tab:chem_desc_organics}
\begin{tabular}{>{\centering\arraybackslash}m{2.5cm} >{\centering\arraybackslash}m{1.5cm} >{\centering\arraybackslash}m{2cm} >{\centering\arraybackslash}m{2cm} >{\centering\arraybackslash}m{2cm}}
\toprule
 & \textbf{SumReal} & \textbf{degraded} & \textbf{intermediate} & \textbf{reference} \\
\midrule
\multirow[t]{2}{*}{\%OC} & mean & 2.110 & 2.405 & 1.779 \\
 & std & 1.599 & 1.458 & 0.682 \\
\cline{1-5}
\multirow[t]{2}{*}{1245-TCB} & mean & 0.906 & 1.201 & 0.555 \\
 & std & 2.321 & 2.143 & 1.035 \\
\cline{1-5}
\multirow[t]{2}{*}{1234-TCB} & mean & 0.252 & 0.234 & 0.253 \\
 & std & 0.257 & 0.240 & 0.332 \\
\cline{1-5}
\multirow[t]{2}{*}{QCB} & mean & 0.729 & 1.255 & 0.636 \\
 & std & 1.015 & 3.055 & 0.871 \\
\cline{1-5}
\multirow[t]{2}{*}{HCB} & mean & 2.759 & 17.713 & 2.904 \\
 & std & 4.291 & 83.487 & 6.011 \\
\cline{1-5}
\multirow[t]{2}{*}{OCS} & mean & 1.213 & 1.502 & 0.721 \\
 & std & 3.395 & 3.606 & 1.874 \\
\cline{1-5}
\bottomrule
\end{tabular}
\end{table}

% Table 3: Pesticides and PCBs
\begin{table}[ht]
\centering
\caption{Descriptive Statistics of Pesticides and PCBs}
\label{tab:chem_desc_pesticides}
\begin{tabular}{>{\centering\arraybackslash}m{2.5cm} >{\centering\arraybackslash}m{1.5cm} >{\centering\arraybackslash}m{2cm} >{\centering\arraybackslash}m{2cm} >{\centering\arraybackslash}m{2cm}}
\toprule
 & \textbf{SumReal} & \textbf{degraded} & \textbf{intermediate} & \textbf{reference} \\
\midrule
\multirow[t]{2}{*}{p,p'-DDE} & mean & 0.679 & 0.485 & 0.324 \\
 & std & 1.255 & 0.930 & 0.328 \\
\cline{1-5}
\multirow[t]{2}{*}{p,p'-DDD} & mean & 3.879 & 0.772 & 0.862 \\
 & std & 14.634 & 1.039 & 0.923 \\
\cline{1-5}
\multirow[t]{2}{*}{mirex} & mean & 0.253 & 0.212 & 0.134 \\
 & std & 0.682 & 0.332 & 0.242 \\
\cline{1-5}
\multirow[t]{2}{*}{Heptachlor Epoxide} & mean & 0.098 & 0.051 & 0.071 \\
 & std & 0.235 & 0.250 & 0.211 \\
\cline{1-5}
\multirow[t]{2}{*}{total PCB} & mean & 15.137 & 10.705 & 7.715 \\
 & std & 32.189 & 36.285 & 16.795 \\
\cline{1-5}
\bottomrule
\end{tabular}
\end{table}

\clearpage
\subsection{Figures}

\begin{figure}[!h]
    \centering
    \includegraphics[width=0.8\textwidth]{../results/preliminary_results/pqrm_quantile_regression_pattern_1.png}
    \caption{Piecewise Quantile Regression for cluster 1}
    \label{fig:pqrm_quantile_regression_pattern_1}
\end{figure}

\begin{figure}[!h]
    \centering
    \includegraphics[width=0.8\textwidth]{../results/preliminary_results/pqrm_quantile_regression_pattern_2.png}
    \caption{Piecewise Quantile Regression for cluster 2}
    \label{fig:pqrm_quantile_regression_pattern_2}
\end{figure}

\begin{figure}[!h]
    \centering
    \includegraphics[width=0.8\textwidth]{../results/preliminary_results/pqrm_quantile_regression_pattern_3.png}
    \caption{Piecewise Quantile Regression for cluster 3}
    \label{fig:pqrm_quantile_regression_pattern_3}
\end{figure}

\begin{figure}[!h]
    \centering
    \includegraphics[width=0.8\textwidth]{../results/preliminary_results/pqrm_pattern_1_tau_0.5.png}
    \caption{Piecewise Quantile Regression for cluster 1 with inference of confidence intervals}
    \label{fig:pqrm_pattern_1_tau_0.5}
\end{figure}

\begin{figure}[!h]
    \centering
    \includegraphics[width=0.8\textwidth]{../results/preliminary_results/pqrm_pattern_2_tau_0.5.png}
    \caption{Piecewise Quantile Regression for cluster 2 with inference of confidence intervals}
    \label{fig:pqrm_pattern_2_tau_0.5}
\end{figure}

\begin{figure}[!h]
    \centering
    \includegraphics[width=0.8\textwidth]{../results/preliminary_results/pqrm_pattern_3_tau_0.5.png}
    \caption{Piecewise Quantile Regression for cluster 3 with inference of confidence intervals}
    \label{fig:pqrm_pattern_3_tau_0.5}
\end{figure}

\begin{figure}[!h]
    \centering
    \includegraphics[width=0.8\textwidth]{../results/preliminary_results/raw_chemical_histgram.png}
    \caption{Example of outlier detection: Raw chemical histogram with IQR-detected outliers in red}
    \label{fig:raw_chemical_histgram}
\end{figure}

% \clearpage
% \subsection{Additional Math background supporting the major methods used in the program}

% \subsubsection{Principal Component Analysis for Sediment Contamination Assessment}

% In SVD, a data matrix $X$ is decomposed as $X = U \Sigma V^T$, where $U$ and $V$ are orthogonal matrices and
%  $\Sigma$ is a diagonal matrix of singular values.
% For PCA, the principal components correspond to
% the directions of maximum variance, which are given by the right singular vectors in $V$.
% By incorporating weights into the data matrix, Weighted PCA modifies the SVD process 
% to emphasize certain observations(row or column), allowing for more flexible dimensionality reduction 
% tailored to the importance of each data point.

% Given a data matrix $X \in \mathbb{R}^{m \times n}$, the SVD decomposes $X$ as $X = U \Sigma V^T$, where:
% \begin{itemize}
%     \item $U \in \mathbb{R}^{m \times m}$ contains the left singular vectors (columns),
%     \item $\Sigma \in \mathbb{R}^{m \times n}$ is a diagonal matrix of singular values,
%     \item $V \in \mathbb{R}^{n \times n}$ contains the right singular vectors (columns).
% \end{itemize}

% To solve for $U$ and $V$:
% \begin{enumerate}
%     \item Compute $X^T X$ and find its eigenvectors and eigenvalues. The eigenvectors form the columns of $V$.
%     \item It can be proven that \(\mu_i^T = \frac{X v_i}{\sigma_i}\) is the $i$-th column of $U$, where $\sigma_i$ is the $i$-th singular value(square root of eigenvalue \(\lambda_i\)).
%     \item The singular values in $\Sigma$ are the square roots of the nonzero eigenvalues from either $X X^T$ or $X^T X$.
% \end{enumerate}

% Mathematically, given a matrix $X$ of shape $(m, n)$, the SVD can be expressed as: 
% \[
% (X^T X) V = V \Lambda, \quad \Lambda = \text{diag}(\lambda_1, \lambda_2, \ldots, \lambda_n)
% \]
% According to spectral theorem, the eigenvalues $\lambda_i$ are non-negative and can be ordered as $\lambda_1 \geq \lambda_2 \geq \ldots \geq \lambda_n \geq 0$.
% \(V\) is an orthogonal matrix, meaning \(V^T V = I\), where \(I\) is the identity matrix.
% It gives:
% \[
% (X^T X) = V \Lambda V^T
% \]
% To a centered sample matrix of size \(m\) with \(n\) features, its covariance matrix is \(\frac{1}{m-1} (X^T X)\).
% Via the eigenvalue decomposition, the variation in it can be expressed by the paris of eigenvalues and eigenvectors,
% which are a series of rank-one matrices that carry different and independent information:
% \[
% \frac{1}{m-1} (X^T X) = \frac{1}{m-1} \sum_{i=1}^{n} \lambda_i v_i v_i^T
% \]

% Therefore, when columns in \(X\) represent different features, the columns \(v_i\) of \(V\) are the principal components 
% that have its values as the linear combinations of the original features, and the scaled eigenvalues \(\frac{\lambda_i}{m-1}\)
% as the amount of variation explained by each principal component.

% Weighted PCA leverages the Singular Value Decomposition (SVD)
% to extract principal components from weighted data.

% \subsubsection{Hierarchical Clustering analysis for Taxa Community Structure Analysis}

% Hierarchical clustering offers a data-driven approach to group sampling sites based on environmental or 
% community-level similarities. Unlike partitioning methods, hierarchical clustering builds a nested tree
% (dendrogram) that captures the progressive grouping of observations based on their dissimilarities.
% In the context of my program, hierarchical clustering can be leveraged to define natural environmental 
% classes prior to modeling the biological response to stressors.

% Let each observation \( x_i \in \mathbb{R}^p \) denote a site with p-dimensional attributes (e.g., environmental variables). 
% The dissimilarity between two observations \( x_i \) and \( x_{i'} \) is measured by a distance function,
% often chosen as the squared Euclidean distance:
% \[ d(x_i, x_{i'}) = \sum_{j=1}^p (x_{ij} - x_{i'j})^2 \]

% Based on the distance metric, we can define the similarity measure on variable and observation levels as the following:
% \[
% \begin{aligned}
%     &\text{Variable level:} \quad d_j(x_{ij}, x_{i'j}) = (x_{ij} - x_{i'j})^2, \quad j = 1, \ldots, p \\
%     &\text{Observation level:} \quad d(x_i, x_{i'}) = \sum_{j=1}^{p} (x_{ij} - x_{i'j})^2 \\
% \end{aligned}
% \]

% To cluster level \( D_C \), we can define other alternatives to the dissimilarity measures, commonly including:

% \[ D_{SL}(G, H) = \min_{i \in G,\; i' \in H} d(x_i, x_{i'}) \quad \text{(Single Linkage)} \]
% \[ D_{CL}(G, H) = \max_{i \in G,\; i' \in H} d(x_i, x_{i'}) \quad \text{(Complete Linkage)} \]
% \[ D_{GA}(G, H) = \frac{1}{|G||H|} \sum_{i \in G} \sum_{i' \in H} d(x_i, x_{i'}) \quad \text{(Average Linkage)} \] 

% \begin{itemize}
%     \item \textbf{Use in the Program:}
%     \begin{itemize}
%         \item Hierarchical clustering is used to categorize sampling sites into clusters with similar environmental conditions before building predictive models linking taxonomic composition to stressor levels.
%         \item This enables a two-stage analysis:
%         \begin{itemize}
%             \item Use environmental variables to form environmental clusters (reference site classification).
%             \item Model stressor-community relationships within or across these clusters to control for natural variation.
%         \end{itemize}
%     \end{itemize}
%     \item \textbf{Advantages:}
%     \begin{itemize}
%         \item Does not require pre-specification of the number of clusters.
%         \item Produces a dendrogram showing how clusters are formed step-by-step.
%         \item Enables ecological interpretation of clusters through tree visualization.
%         \item Allows separation of natural variation from anthropogenic stress impacts.
%     \end{itemize}
%     \item \textbf{Model-Based Interpretation:}
%     \begin{itemize}
%         \item Assume that each cluster \( k \) corresponds to a latent distribution \( p_k(x) \), with the overall mixture model:
%         \[
%         p(x) = \sum_{k=1}^K \pi_k p_k(x)
%         \]
%         \item Each observation is generated as \( x \sim p_k(x) \) conditional on cluster membership \( k \), providing a statistical grounding for hierarchical clustering in unsupervised structure discovery.
%     \end{itemize}
% \end{itemize}


% \subsubsection{Piecewise Quantile Regression for Threshold Determination}

% Let $m_\tau(x; \boldsymbol{\beta}_\tau, \boldsymbol{\alpha}_\tau)$ be the $\tau$th quantile of the conditional distribution 
% of the ecological response given environmental condition. Then we define:

% \[
% y_\tau = m_\tau(x; \boldsymbol{\beta}_\tau, \boldsymbol{\alpha}_\tau) + \varepsilon_\tau.
% \]

% The form is similar to the linear regression model, but there are no restrictions on the distribution of error term $\varepsilon_\tau$, 
% which means the error terms can be heteroscedastic(non-constant variance).


% Then the \textbf{PQRM} with two breakpoints is defined as:

% \[
% m_\tau(x_i; \boldsymbol{\beta}_\tau, \boldsymbol{\alpha}_\tau) =
% \begin{cases}
% \beta_{0\tau} + \beta_{1\tau} x_i & \text{for } x_i \leq \alpha_{1\tau} \\
% \beta_{0\tau} + \beta_{1\tau} x_i + \beta_{2\tau}(x_i - \alpha_{1\tau}) & \text{for } \alpha_{1\tau} < x_i \leq \alpha_{2\tau} \\
% \beta_{0\tau} + \beta_{1\tau} x_i + \beta_{2\tau}(x_i - \alpha_{1\tau}) + \beta_{3\tau}(x_i - \alpha_{2\tau}) & \text{for } x_i > \alpha_{2\tau}
% \end{cases}
% \tag{3}
% \]

% The \(\tau\) subscript indicates the quantile level.
% Breakpoints are spotted in the predictor space, which seems uncorrelated with the response variable and its quantile,
% but they are not fixed and should be estimated under different quantile levels so that
% the piecewise model provides a better performance on the data with such quantile level.
% That is, the breakpoints are also functions of the quantile level \(\tau\).
% \[
% Y_{\tau} = X(\alpha_{_({\tau}; 1)}, \alpha_{_({\tau}; 2)}) \cdot \boldsymbol{\beta_{\tau}} + \boldsymbol{\varepsilon_{\tau}}
% \]

% Because the model are not predicting the means but the quantiles of $y$'s at given $x$'s,
% the measurement for the model performance should no longer be the mean squared error(MSE),
% \footnote{If a model regresses the mean of $Y$, it should outperform other models in the measurement of MSE, assuming the assumptions are satisfied.}
% the check function is the loss function, which is defined as:
% \[
% \rho_\tau(u) =
% \begin{cases}
% \tau u & \text{if } u \geq 0 \\
% (\tau - 1) u & \text{if } u < 0
% \end{cases}
% \quad \text{where } u = y - \hat{y}
% \]

% The loss function is a random variable that derivatives from $Y$, and it also can be viewed as a function of 
% the prediction $\hat{y_{\tau}}$, which is not a RV.
% A specific quantile can be found by minimizing the expected loss function - $E({\rho_{\tau}(Y - \hat y_{\tau})})$ with respect to $\hat y_{\tau}$ across
% the observations:
% \[
%     {\displaystyle q_{Y}(\tau )={\underset {\hat y_{\tau}}{\mbox{arg min}}}E(\rho _{\tau }(Y-\hat y_{\tau}))={\underset {\hat y_{\tau}}{\mbox{arg min}}}{\biggl \{}(\tau -1)\int _{-\infty }^{\hat y_{\tau}}(y-\hat y_{\tau})dF_{Y}(y)+\tau \int _{\hat y_{\tau}}^{\infty }(y-\hat y_{\tau})dF_{Y}(y){\biggr \}}} 
% \]

% The expectation is no more a random variable, but a function of $(\hat y_{\tau}, y, F_Y(y))$.
% By computing the derivative of it with respect to $\hat y_{\tau}$, the equation is determined by $(\hat y_{\tau}, F_Y(y))$.
% Setting it to zero and letting $q_{\tau}$ be the solution of $\hat y_{\tau}$ to the equation, we have:

% \[
%     {\displaystyle 0=(1-\tau )\int _{-\infty }^{q_{\tau }}dF_{Y}(y)-\tau \int _{q_{\tau }}^{\infty }dF_{Y}(y).} 
% \]

% It gives:
% \[
% F_Y(q_{\tau}) = \tau
% \]

% Therefore, $\rho_\tau(u)$ is a valid loss function for inferring the quantile of $Y$.
% However, when building quantile regression on $Y$ with $X$, we need to have enough observations on each condition of $X$.
% Because there is no assumption that residuals are homoscedastic, the globally minimized loss function does not necessarily 
% bring good predictions on all conditions, which is different to the linear regression model.

% To a given condition $x$, the quantile of $\tilde y$ is defined as:
% \[
%     {\displaystyle q_{\tilde y|x}(\tau )={\underset {\hat y_{\tau}}{\mbox{arg min}}}E(\rho _{\tau }(\tilde y - \hat y_{\tau})|X)={\underset {\hat y_{\tau}}{\mbox{arg min}}}{\biggl \{}(\tau -1)\int _{-\infty }^{\hat y_{\tau}}(y - \hat y_{\tau})dF_{\tilde y|x}(y)+\tau \int _{\hat y_{\tau}}^{\infty }(y-\hat y_{\tau})dF_{\tilde y|x}(y){\biggr \}}} 
% \]

% The expected loss function that provides quantiles over all observations is:
% \[
%     E(\rho_{\tau}(\tilde y - \hat y_{\tau})) = \frac{1}{n} \sum_{i=1}^n \rho_\tau(\tilde y_i - x_i^\top \beta_{\tau})
% \]

% Optimizing it to minimum with respect to $\beta_{\tau}$ gives the optimal $\beta_{\tau}$
% \footnote{There should be an assumption of linear relation on the amount changing between $x$ and $\tau$ quantile of $y$, and $\beta$ is the coefficient of that relation.}:
% \[
% \hat{\beta}\tau = {\underset {\beta_{\tau}}{\mbox{arg min}}} E(\rho_{\tau}(\tilde y - \hat y_{\tau}))
% \]



% When given two breakpoints, the $(\alpha_{(\tau; 1)}, \alpha_{(\tau; 2)})$ can be iteratively searched by 
% Newton-Raphson method. Within each iteration, the loss function is minimized to find the optimal $\beta_{(\tau)}$ vector.

% \subsubsection{Synthetic Data Generation}

% Synthetic data generation refers to the process of artificially creating data that mimics 
% the statistical properties and structure of real-world datasets. 
% The core motivation is to generate data when real data are limited, imbalanced,
%  private, or costly to obtain. Synthetic data are widely used in machine learning,
%   data privacy preservation, and simulation-based research.

% This approach can simulate various data types, 
% including tabular (structured), image, text, and time series data. 
% The generated data should reflect similar distributions, correlations,
%  and interactions as the original data while maintaining flexibility and scalability.

% Synthetic data generation can play a supportive role in this study by addressing several data-related challenges 
% commonly encountered in ecological assessments:

% \begin{itemize}
%     \item \textbf{Enhancing model robustness}: By generating additional synthetic observations that mimic the real data structure, we can augment the existing data pool, which helps reduce model overfitting and improve generalization when predicting stressor impacts across unsampled sites.
    
%     \item \textbf{Balancing site distribution across gradients}: Many ecological datasets are imbalanced with respect to environmental gradients or stressor intensity levels. Synthetic data can help balance the representation of different ecological zones or stressor conditions, ensuring the model learns from a more evenly distributed set of scenarios.
    
%     \item \textbf{Supporting rare condition modeling}: Stressor levels or taxa responses in extreme or under-sampled regions (e.g., highly degraded or pristine sites) can be underrepresented. Synthetic data can simulate these rare cases to aid in threshold detection or to inform prediction in sparsely observed domains.
% \end{itemize}

% A simple yet effective approach to synthetic data generation is random sampling (bootstrapping) from the observed dataset.
%  Given a dataset \( \mathcal{D} = \{x_1, \dots, x_N\} \), where \( x_i \in \mathbb{R}^p \), 
%  synthetic samples are generated by sampling with replacement:
% \[
% x_i^* \sim \hat{F}, \quad i = 1, \dots, N^*
% \]
% where \( \hat{F} \) is the empirical distribution of \( \mathcal{D} \), and \( N^* \) is
%  the desired number of synthetic samples.

% This method preserves the multivariate structure by sampling entire rows, maintaining correlation between features.
%  It is non-parametric and computationally efficient.

% \textbf{Use in this project:} Random sampling can augment small or imbalanced training sets, support bootstrapped 
% threshold estimation, and improve model robustness across environmental gradients.



\end{document}

